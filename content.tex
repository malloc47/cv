%% Shared

\newcommand{\statueOfLiberty}{
  \raisebox{-.4ex}{\protect\includegraphics[height=2.5ex]{img/statue-of-liberty-emoji.png}}}

\newcommand{\chicagoFlag}{
  \raisebox{-.7ex}{\protect\includegraphics[height=2.5ex]{img/chicago-flag.png}}}

\NewDocumentCommand\itology{s}{%
  \IfBooleanTF{#1}{%
    {\href{https://www.it-ology.org/}{IT-oLogy}\xspace}}%
    {IT-oLogy\xspace}}

\NewDocumentCommand\usc{s}{%
  \IfBooleanTF{#1}{%
    {\href{https://sc.edu/}{UofSC}\xspace}}%
    {UofSC\xspace}}

\NewDocumentCommand\uscl{s}{%
  \IfBooleanTF{#1}{%
    {\href{https://sc.edu/about/system_and_campuses/lancaster/index.php}{USCL}\xspace}}%
    {USCL\xspace}}

%% Intro

\newcommand{\name}{
  \par{\centering {\Huge Jarrell \textsc{Waggoner}}\bigskip\par}}

\newcommand{\biographical}{
  \begin{center}
    \statueOfLiberty{} |
    \href{https://www.malloc47.com}{malloc47.com} |
    {\sans \raisebox{-.2ex}{✉}} \href{mailto:jarrell.waggoner@gmail.com}{\color{black}{jarrell.waggoner@gmail.com}} |
    {\sans ☏} \href{tel:847-261-4747}{\color{black}{847-261-4747}} |
    \{ \begin{small}
      \href{https://www.github.com/malloc47}{github.com/},
      \href{https://www.linkedin.com/in/malloc47}{linkedin.com/in/},
      \href{https://twitter.com/malloc47}{@}\end{small}
    \}
    \textbf{malloc47}
  \end{center}
}

\newcommand{\interests}{computer vision, image processing, artificial
  intelligence, pattern recognition \& machine learning, data science,
  data engineering, functional programming, web development, GIS,
  Clojure}

\newcommand{\interestsLong}{computer vision, segmentation, document
  image analysis, event recognition, image processing, artificial
  intelligence, pattern recognition \& machine learning, data science,
  functional programming, GIS, Clojure}

\newcommand{\summary}{Engineering leader with decades-long career
  spanning research science, early-stage startups, and Fortune 5
  companies. Specializing in the intersection of data science and data
  engineering with extensive experience architecting MLOps tools and
  data platforms. Provide technical leadership across an org-level
  portfolio of teams/projects while staying hands-on developing
  backend systems, with particular emphasis on functional languages:
  10~years of experience using Clojure to build data pipelines,
  microservices, and distributed systems.}

%% Education

\newcommand{\phdDegree}{
  \degree{08/2013}%
  {Ph.D.}%
  {Doctor}%
  {of}%
  {Computer Science \& Engineering}%
  {University of South Carolina}%
  {Dr. Song \textsc{Wang}}%
  {Multi-Label Segmentation Propagation for Materials Science Images
    Incorporating Topology and Interactivity}%
  {}}

\newcommand{\meDegree}{
  \degree{05/2009}%
  {M.E.}%
  {Master of Engineering}%
  {in}%
  {Computer Science}%
  {University of South Carolina}%
  {}%
  {}%
  {\small\emph{magna cum laude}}}

%% Experience

\newcommand{\optum}{
  \experience{\href{https://www.optum.com/}{Optum, Inc.}}%
  {Software Architect \textcolor{lightg}{(Grade 30)}}%
  {Remote (NYC)}%
  {2021---Present}%
  {Brought in through acquisition of Rally Health, continuing as
    Architect of the former Rally data organization now housed within
    the rebranded \textit{Optum Digital} entity.

    \begin{exlist}

    \item Serve as \textit{Domain Workgroup Representative}
      responsible for curating the technologies used by the data and
      ML practices in Optum's centralized technology governance model
      with influence across all of the Optum Tech organization.

    \item Lead an \textit{Architecture Advocates} workgroup embedding
      architecture-focused engineers on individual teams, fostering a
      cross-cutting, big-picture view of our architecture among our
      data teams.

    \item Built a crawler to extract table-level data lineage from our
      \airflow DAGs, generating a 2000 node graph navigable through a
      custom-built \js graph interface.

    \end{exlist}}}

\newcommand{\rally}{
  \experience%
  {\href{https://www.rallyhealth.com}{Rally Health, Inc.}}%
  {Software Architect}%
  {Remote (NYC)}%
  {2019---2021}%
  [Principal Software Engineer]%
  [Minneapolis, MN]%
  [2017---2019]%
  [\textcolor{medg}{\begin{small}(Acquired by Optum)\end{small}}]%
  {Staff-level engineering leader responsible for overseeing technical
    design and engineering decisions across Rally's entire data
    organization that reached a peak of 90+ engineers, analysts,
    managers, and data scientists.

    \begin{exlist}
    \item Brought onboard while Rally was a small startup to architect
      a complete rewrite of the data platform for the whole company,
      moving from a fixed \skill{Cloudera} cluster to a self-service
      platform using \skill{Databricks} and \skill{Redshift} fed by
      \spark ETLs written in \scala and scheduled with \airflow atop
      \skill{Kubernetes}. Built consensus on new architecture and
      delivered working system within a year.

    \item Member of the Rally Engineering Technical Staff, responsible
      for making cross-cutting engineering decisions, evaluating
      potential acquisitions, managing the RFC process, and organizing
      technical interest groups.

    \item Heavily involved in defining team structure and hiring,
      conducting over 140 interviews for IC and management roles to
      scale the data organization from 4 data engineers to over 50
      data engineers.

    %% \item Fostered good design practices across the data organization
    %%   by auditing design documents, leading architecture meetings for
    %%   multiple teams to grow design aptitude, and spearheading the
    %%   standardization of best-practices on the team including code
    %%   review, documented on-call workflow, CI/CD ,
    %%   monitoring/alerting, idempotent ETL workflows, and reproducible
    %%   EDW layout.

    \item Coordinated with over 16 internal and external teams and
      vendors across an extensive range of projects including
      productionalized ML workflows, frontend/mobile event tracking,
      real time data processing, data anonymization,
      security/compliance/privacy requirements, data ingestion APIs,
      data quality validation, and self-service internal product
      analytics.

    \end{exlist}}}

\newcommand{\drw}{
  \experience{\href{https://www.drw.com}{DRW Holdings, LLC}}%
  {Software Engineer}%
  {Chicago, IL}%
  {2016---2017}%
  {Member of the Trading Infrastructure team, developing the internal
    platform used by every trading desk at DRW. Built greenfield
    high-performance service-oriented systems using \clojure and \java
    while maintaining legacy applications in \ruby and \csh among a
    catalog of over 50 microservices.

    \begin{exlist}
    \item Contributed to a \ruby-based reconciliation tool used to
      balance cash flows for high-volume trading.

    \item Extended a graph-based research workflow tool used for
      computing the value and settle price of options, futures,
      equities, and other financial instruments, written in \clojure.

    \item Developed and extended multiple UI frontends for internal
      tools using \react and \reagent.
  \end{exlist}}}

\newcommand{\groupon}{
  \experience{\href{https://www.groupon.com}{Groupon, Inc.}}%
  {Senior Software Engineer \textcolor{lightg}{(SDE IV)}}%
  {Chicago, IL}%
  {\textcolor{lightg}{02/}2016---\textcolor{lightg}{09/}2016}%
  [Software Engineer \textcolor{lightg}{(SDE III)}]%
  []%
  [\textcolor{lightg}{08/}2013---\textcolor{lightg}{02/}2016]%
  {Contributed to multiple teams solving cross-cutting data
    engineering and \skill{MLOps} problems.
    \begin{exlist}
      \item \textbf{Flux team}: Responsible for a data pipeline
        management and machine learning platform used to run
        productionalized decision tree learning models to predict
        customer attrition, lifetime customer value, and merchant
        value. Spearheaded implementation of distributed systems for
        the feature store, job scheduling, and data catalog components
        used by all models on the platform. Fed from \teradata,
        written in \clojure, and backed by \hive.

      \item \textbf{Project Genesis strike team}: Coordinated with
        product and business teams to build an ETL to inject 250K
        leads in \salesforce from scraped web data that increased the
        unassigned leads pool by 10X.

      \item \textbf{Supply Intelligence team}: Tech lead overseeing
        the critical business automation of lead-to-salesperson
        assignment that previously required \~{}80 sales managers to
        conduct manually; led the effort to rearchitect this legacy
        system from an ad-hoc job scheduling platform written in \ruby
        and \bash to a multi-staged \hadoop pipeline written in
        \clojure allowing it to scale to 6M daily candidate
        assignments.

      %% \item Built a \postgres-backed high-performance caching and
      %%   write management system in \clojure around the \salesforce API
      %%   that hits 10K req/min.

      %% \item Mentored interns and junior developers, established best
      %%   practices, served as reviewer for company-wide RFCs, and led
      %%   multiple major technical initiatives on a team of 5
      %%   developers.

      %% \item Operationalized \python within the organization by
      %%   spinning up an internal PyPI server; introducing a pex-driven
      %%   single-artifact deployment process; creating an interest
      %%   group, mailing list, and internal wiki; and standardized
      %%   configuration, logging, and resource management.

    \end{exlist}}}

\newcommand{\terrastride}{
  \experience{\href{http://www.terrastride.com/}{TerraStride, Inc.}}%
  {Technical Engineer}%
  {Columbia, SC}%
  {2012---2014}%
  {Founding developer in an agile startup environment creating the
    \href{https://www.huntstand.com}{huntstand.com} web application.
    Written using \python, \django, and \backbone; deployed to
    \skill{AWS}. Responsible for curating full technology stack and
    coordinating with $5$ developers.}}

\newcommand{\palmettocomputerlabs}{
  \experience{\href{https://web.archive.org/web/20190716230447/http://palmettocomputerlabs.com/}{Palmetto Computer Labs}}%
  {Project Manager}%
  {Columbia, SC}%
  {2011---2013}%
  {Created and taught workshops on \git, the \linux command line,
    \android development, and open source software for hundreds of
    students, developers, and government officials at \itology*.
    Managed the
    \href{https://web.archive.org/web/20131104022533/http://open-it-lab.com/}{Open
      IT Lab} and associated projects. Assisted in planning
    \href{https://posscon.org/}{POSSCON}.}}

\newcommand{\elasticvision}{
  \experience{Elastic Vision Consulting}%
  {Contractor}
  {Columbia, SC}%
  {2011}%%
  {Built a parser and generator for \skill{XML} medical records
    formats (CCR and CCD) in a \java web application.  Written using
    \skill{JDOM}, \skill{Xerces}, and \skill{Hibernate}, and run on an
    \skill{Axis2+Jetty6} driven server.}}

\newcommand{\cse}{
  \experience{\href{https://www.cse.sc.edu/}{\usc Department of Computer Science
      and Engineering}}%
  {Teaching Assistant}%
  {Columbia, SC}%
  {2007---2011}%
  {Taught classes in software development, web development, and
    computer engineering, covering \java, \js, \html, and
    \skill{Visual Basic}.  Created syllabi and course objectives,
    developed and graded projects and assignments, supervised labs,
    and tutored students.}}

\newcommand{\internship}{
  \experience{JAARS, Inc.}%
  {Intern---Technical Writer}%
  {}%
  {2005}%
  {Created documentation and integrated context-sensitive online help
    system for speech and linguistic software written in C++ and
    Visual Basic.}}

\newcommand{\volunteering}{
  \experience{JAARS, Inc.}%
  {Volunteer Software Developer}%
  {}%
  {2001---2002}%
  {Spearheaded the conversion from \skill{Visual Basic 4} to
    \skill{Visual Basic 6} for the linguistic reference tool
    \href{https://web.archive.org/web/20170614091024/http://www-01.sil.org/computing/ipahelp/ipaprvw2.htm}{IPA
      Help}.
  {\link{Code}{https://github.com/malloc47/ipa-help}}}}

\newcommand{\afosr}{
  \academic{\usc \href{https://cvl.cse.sc.edu/}{Computer Vision Lab}}%
  {Research Assistant}%
  {funded by \href{https://www.afrl.af.mil/AFOSR/}{AFOSR}}%
  {Materials Volume Segmentation}%
  {Columbia, SC}%
  {2011---2013}%
  {Conducted dissertation research on segmentation methods to extract
    important physical characteristics from image volumes of metallic
    and biologic materials, developed using \python, \numpy, \scipy,
    \opencv, and \matlab. Managed the lab computer network and
    organized weekly lab meetings. Created a \django{}-based web
    application for manual segmentation, an ML-trained classifier for
    assisted segmentation, and a fully automated energy-minimization
    segmentation approach, with large-scale evaluations on real and
    synthetic datasets.}
  [\linkf{Code}{https://www.github.com/malloc47/matsciseg}]
}

\newcommand{\darpa}{
  \academic{DARPA
    \href{https://en.wikipedia.org/wiki/Mind\%27s_Eye_(US_military)}{Mind's
      Eye Program} }%
  {Research Assistant}%
  {funded by \href{https://www.darpa.mil/}{DARPA}}%
  {Video Event Recognition}%
  {Columbia, SC}%
  {2010---2011}%
  {Built video event recognition systems for the DARPA
    \href{https://en.wikipedia.org/wiki/Mind\%27s_Eye_(US_military)}{Mind's Eye Program},
    collaborating with $10$ students and faculty members across three
    institutions to create an AI system that describes events in a
    video clip as natural language sentences.
    %% Attended P.I. meetings in San Diego (2010) and Colorado (2011).
    Developed algorithms in \scheme, \bash, \matlab, and \c to process
    a corpus of 3480 videos extracted into over 1.5 million
    frames. Conducted distributed processing on the
    \href{https://www.top500.org/system/176136/}{Steele} cluster which
    was, at the time, among the top 500 most powerful supercomputing
    clusters.}
  [\linkf{Website}{https://0xab.com/research/video-events.html}
   \linkf{Code}{https://www.github.com/malloc47/video-in-sentences-out}]}

\newcommand{\nehfellow}{
  \academic{\usc \href{https://web.qa.sc.edu/about/centers/digital_humanities/index.php}{Center for Digital Humanities}}%
  {NEH Fellow}%
  {at the \href{https://web.qa.sc.edu/about/centers/digital_humanities/index.php}{Center for Digital Humanities}}%
  {Digital Collation}%
  {Columbia, SC}%
  {2009---2010}%
  {Developed the prototype for a \emph{digital collation} application
    as part of the
    \href{https://web.archive.org/web/20140104133805/http://sapheos.org/}{Sapheos}
    /
    \href{https://web.qa.sc.edu/about/centers/digital_humanities/projects/paragon.php}{Paragon}
    project to identify sub-textual inconsistencies among multiple
    scanned copies of \emph{The Faerie Queene} by \textsc{Edmund
      Spenser}.  Created in \matlab using \opencv to process tens of
    thousands of book page images.}
  [\link{Paper}{https://securegrants.neh.gov/PublicQuery/main.aspx?f=1&gn=HD-50880-09}
   \link{Narrative}{https://thinkingtogether.org/rcream/dossier/sapheosnarrative.pdf}
   \linkf{Code}{https://www.github.com/malloc47/digital-collation}]}

\newcommand{\crayton}{
  \academic{}%
  {\href{https://web.archive.org/web/20190618084356/http://www.gk12.org/}{GK-12} Fellow at \textsc{Crayton Middle School}}%
  {}%
  {8\textsuperscript{th} Grade Science}%
  {Columbia, SC}%
  {2008---2009}%
  {Served in Crayton Middle School, coordinating with the classroom
    instructor to enhance the STEM curriculum and activities in an
    8\textsuperscript{th} grade science classroom. Subsequently
    coordinated and taught at the \textsc{GK-12 Institute for
      Teachers}, presenting the activities developed and delivered in
    the classroom.}}

\newcommand{\cseTA}{
  \academic{}%
  {Graduate Teaching Assistant at \usc*}%
  {}%
  {Web Development}%
  {Columbia, SC}%
  {2007---2008, 2011}%
  {Supervised CSCE~145 labs, covering software development with
    \textsc{Java}, and taught CSCE~102, covering \textsc{Javascript},
    \textsc{HTML}, and \textsc{CSS}. Taught~CSCE~211 covering digital
    logic design.}}

\newcommand{\usclTA}{
  \academic{}%
  {Instructor for \textsc{CSCE 204} at \uscl*}%
  {}%
  {Introductory  Programming}%
  {Lancaster, SC}%
  {Spring 2007}%
  {Hired as special faculty. Taught introductory Visual Basic for
    majors and non-majors. Selected textbooks, developed all course
    material, graded all assignments. Worked with Dr. Noni
    M. Bohonak}}

\newcommand{\campInstructor}{
  \academic{}%
  {Camp Instructor for \textsc{\uscl* Arts and Sciences Adventure Camp}}%
  {}%
  {5\textsuperscript{th}-8\textsuperscript{th} Grade Students}%
  {Lancaster, SC}%
  {Fall 2006}%
  {Worked in collaboration with Dr. Dwayne Brown. One of two
    instructors teaching Math and Computer Science to grade school
    students.}}

\newcommand{\tutor}{
  \academic{}%
  {Professional Tutor at \textsc{\uscl* Academic Success Center}}%
  {}%
  {High School and College Students}%
  {Lancaster, SC}%
  {2003---2007}%
  {Student and graduate tutor for college-level Mathematics, Computer
    Science, Physics, and English classes.}}

%% Skills

\newcommand{\skillsList}{
  \skills{\clojure}{\threeskill}
  \skills{\python}{\threeskill}
  \skills{\java}{\threeskill}
  \skills{\git}{\threeskill}
  \skills{\linux}{\threeskill}
  \skills{\postgres}{\threeskill}
  \skills{\airflow}{\threeskill}
  \skills{\hadoop ecosystem}{\threeskill}
  \skills{\spark}{\threeskill}
  \skills{\js}{\twoskill}
  \skills{\scala}{\twoskill}
  \skills{\skill{Nix/NixOS}}{\oneskill}
}

\newcommand{\skillsListLong}{
  \skills{\clojure}{\threeskill}
  \skills{\clojurescript}{\twoskill}
  \skills{\python}{\threeskill}
  \skills{\bash}{\twoskill}
  \skills{\java}{\threeskill}
  \skills{\ccpp}{\twoskill}
  \skills{\js}{\twoskill}
  \skills{\haskell}{\oneskill}
  \skills{Emacs Lisp}{\twoskill}
  \skills{\LaTeX}{\oneskill}
  \skills{\matlab}{\twoskill}
  \skills{\scala}{\twoskill}
  \skills{\scheme}{\threeskill}
  \skills{\ruby}{\twoskill}
  \skills{\git}{\threeskill}
  \skills{\linux}{\threeskill}
  \skills{\django}{\twoskill}
  \skills{\react / \reframe}{\oneskill}
  \skills{\postgres}{\threeskill}
  \skills{\airflow}{\threeskill}
  \skills{\begin{footnotesize}\hadoop ecosystem\end{footnotesize}}{\threeskill}
  \skills{\numpy/\scipy}{\threeskill}
  \skills{\opencv}{\threeskill}
  \skills{\spark}{\threeskill}
}

\newcommand{\skillsLegend}{
  \begin{footnotesize}
    \oneskill Small-scale or personal projects \hfill
    \twoskill Used in production  \hfill
    \threeskill Used in large-scale production systems
  \end{footnotesize}}

% graveyard

%% \skills{Blender}{\twoskill}
%% \skills{English}{\threeskill}
%% \skills{LAMP Stack}{\fournotes}
%% \skills{LISP}{\onenote}
%% \skills{Learning}{\fournotes Machine}
%% \skills{MS Office}{\fivenotes}
%% \skills{Maple}{\twoskill}
%% \skills{Networking}{\threenotes}
%% \skills{Processing}{\fivenotes Image}
%% \skills{Sys. Admin.}{\threenotes}
%% \skills{Visual Basic}{\fivenotes}
%% \skills{Windows}{\fivenotes}
%% \skills{Wordpress}{\fournotes}
%% \skills{\hive}{\oneskill}
%% \skills{\html}{\threeskill}
%% \skills{\jquery}{\twoskill}
%% \skills{\php}{\oneskill}


%% Projects

\newcommand{\matsciseg}{
  \proj{matsciseg}%
  {Framework for propagated 3D volume segmentation, used in my
  dissertation work.  Algorithms created in \python and \cpp and
  exposed as a web API using \django. Includes a web application that
  consumes the API created in \js, and \jquery.}%
  {github.com/malloc47/matsciseg}}

\newcommand{\nonpartisanme}{
  \proj{\href{http://nonpartisan.me}{nonpartisan.me}}%
  {Google Chrome extension that filters social media websites for
  political keywords.  Available in the
  \href{https://chrome.google.com/webstore/detail/nonpartisanme/ninebcppidndhampaggnjbijpacoadgg}{Chrome
    Web Store}.  Featured in the
  \href{https://charlestoncitypaper.com/sick-of-politics-on-facebook-try-this-browser-tool/}{Charleston City Paper}.}%
  {github.com/malloc47/nonpartisan.me}}

\newcommand{\termdo}{
  \proj{term-do}%
  {An interactive terminal prompt that displays potential command
    completions as you type.  A hybrid of gnome-do and Emacs's
    ido-mode.  Works on many tested VT100 terminal types; built
    in~\cpp. Includes client/server architecture implemented
    with \skill{boost.interprocess} and a plugin system.}%
  {github.com/malloc47/term-do}}

\newcommand{\ratiocontour}{
  \proj{Ratio Contour}%
  {Maintainer and contributor for the Ratio Contour project, a salient
  object detection and segmentation method used for computer vision
  applications.  Developed in \c and \matlab.}%
  {github.com/malloc47/ratio-contour}}

\newcommand{\digitalcollation}{
  \proj{Digital Collation}%
  {Research project to ``collate'' high-resolution documents by using
    image registration, accomplished using the \skill{SIFT} feature
    detector and a
    \href{https://en.wikipedia.org/wiki/Thin_plate_spline}{thin plate
      spline} warping technique, written in \matlab.}%
  {github.com/malloc47/digital-collation}}

\newcommand{\pmldap}{
  \proj{PMLDAP}%
  {\linux user management tool for Linux clusters.  Created as a
    simplified replacement for \skill{LDAP}.  Capable of bootstrapping
    new systems, synchronizing users and configuration files, and
    running distributed commands.  Written in \bash.}%
  {github.com/malloc47/pmldap}}

\newcommand{\matscicut}{
  \proj{matscicut}%
  {An energy minimization framework for segmenting 3D materials
    volumes. Prototype of dissertation work, created in \cpp using
    \opencv libraries, with assorted \matlab helper utilities.}%
  {github.com/malloc47/matscicut}}

\newcommand{\githq}{
  \proj{git-hq}%
  {A remote management system for \git, created in \python.}%
  {github.com/malloc47/git-hq}}

\newcommand{\sinaweibo}{
  \proj{Sina Weibo Mobile Client}%
  {Created a \skill{J2ME}-based prototype mobile client for the
    popular Chinese Sina microblogging service, similar to Twitter.
    Targeted at limited-functionality CLDC phones and uses a custom
    \skill{Java} wrapper for the Sina API.  Employs symmetric-key
    encryption for personal data.}%
  % TODO: dead link
  {bd.weibo.10086.cn/2012/downloads\_kjav}}

\newcommand{\befungepy}{
  \proj{befunge.py}%
  {Complete \href{https://en.wikipedia.org/wiki/Befunge}{Befunge}
    interpreter written in \python.  Implements the Befunge 93
    specification, and is one of the closest Python equivalents to the
    \c reference implementation.}
  {github.com/malloc47/befunge.py}}

\newcommand{\cockpit}{
  \proj{cockpit}%
  {Single-page web application dashboard deplayed on a wall-mounted
    tablet that shows the local time, weather, and transit departure
    times. Written in \clojurescript using the \reframe framework and
    styled using the \href{https://mui.com/}{MUI} \react toolset.
    \\
    \linkf{Blog Series}{https://www.malloc47.com/building-a-personal-dashboard-in-clojurescript}}
  {github.com/malloc47/cockpit}
}

%% References

\newcommand{\referenceslist}{
  \nocite{derrick:16}
  \nocite{waggoner:15}
  \nocite{zhou:14}
  \nocite{waggoner:phd}
  \nocite{waggoner:14}
  \nocite{waggoner:13a}
  \nocite{waggoner:13c}
  \nocite{waggoner:11}
  \nocite{wang:11}
  \nocite{temlyakov:10}
  \nocite{zhang:10}
  \nocite{waggoner:12}
  \nocite{barbu:12}
  \nocite{barbu:12b}
  \nocite{zhang:12}
  \nocite{temlyakov:13}
  \nocite{salvi:13a}
  \nocite{salvi:13b}
}

\newcommand{\referencesShort}{
  \nocite{derrick:16}
  \nocite{waggoner:15}
  \nocite{waggoner:phd}
  \nocite{waggoner:14}
  \nocite{waggoner:13a}
  \nocite{barbu:12}
  \nocite{waggoner:11}
}

\newcommand{\presentations}{
  \nocite{uhg:19}
  \nocite{posscon:15}
  \nocite{ato:13}
  \nocite{gsd:13}
  \nocite{posscon:13}
  \nocite{uscl:13}
  \nocite{spie:13}
  \nocite{appathon:12}
  \nocite{uscsymposium:12}
  \nocite{scmta:12}
  \nocite{sctcs:12}
  \nocite{dhhpc:12}
  \nocite{spie:12}
  \nocite{scgmis:12}
  \nocite{darpa:11}
}

\newcommand{\presentationsShort}{
  \nocite{uhg:19}
  \nocite{posscon:15}
  \nocite{ato:13}
  \nocite{posscon:13}
}

\newcommand{\referencesPlusPresentations}{
  \nocite{uhg:19}
  \nocite{derrick:16}
  \nocite{waggoner:15}
  \nocite{posscon:15}
  \nocite{waggoner:phd}
  \nocite{waggoner:14}
  \nocite{waggoner:13a}
  \nocite{barbu:12}
  \nocite{waggoner:11}
}

%% Guest Lectures

%% \begin{enumerate}
%% \renewcommand{\labelenumi}{[G\arabic{enumi}] }
%% %% \item \emph{Combining Global Labeling and Local Relabeling for Metallic Image Segmentation}. Graduate Student Day Competition, Second Place. April 8, 2011.
%% %% \item \emph{Image Registration for Digital Collation}. Graduate Student Day Competition, Honorable Mention. April 2, 2010.
%% \item \emph{Building Chrome Extensions}.  In CSCE 242.  Guest lecture for Dr. José M. Vidal.  November~30, 2012.
%% \item \emph{Modeling in Blender}.  In CSCE 552.  Guest lecture for Dr. Jijun Tang.  February~28, 2011.
%% \item \emph{Aspect-Oriented Programming}. In CSCE 531. Guest lecture for Dr. Marco Valtorta. March 19, 2008.
%% \item \emph{Math 241}. Vector Calculus. Guest lecture for Dr. Dwayne Brown. April~23---26, 2007.
%% \item \emph{Math 242}. Differential Equations. Guest lecture for Dr. Dwayne Brown. April~23---26, 2007.
%% \end{enumerate}

%% Travel

%\begin{enumerate}
%\renewcommand{\labelenumi}{[T\arabic{enumi}] }
%\item \emph{Mind's Eye PI Meeting}. DARPA Project P.I.~meeting. Denver, CO. January 20---21, 2011.
%\item \emph{Tomography and its Applications to Materials Science and Non-Destructive Evaluation}. Organizd by M. De Graef, L. Drummy, J. Simmons, M. Comer, C. Bouman, and J. Knopp. Tech\^{}Edge, Dayton, Ohio. December 13---15, 2010.
%\item Visiting scholar. In collaboration with J. M. Siskind. Purdue University, West Lafayette, IN. December 6---22, 2010 \& January 5---16, 2011.
%\item \emph{Mind's Eye Kickoff Meeting}. DARPA Project P.I.~meeting. San Diego, CA. September 23---24, 2010.
%\end{enumerate}


%% Honors/Awards

\newcommand{\awardlist}{
  \award{First Place}{Gamecock Computing Research Symposium Poster Session}{\usc*}{2012}
  \award{First Place}{Graduate Student Day Presentation}{\usc}{2012}
  \award{Second Place}{Graduate Student Day Presentation}{\usc}{2011}
  \award{Honorable Mention}{Graduate Student Day Presentation}{\usc}{2010}
  \award{Inductee}{Upsilon Pi Epsilon}{\usc}{2009}
  \award{Recipient}{Clara P. Hammond Award}{\uscl*}{2004}
  \award{Recipient}{Science and Mathematics Award}{\uscl}{2004}
  \award{Recipient}{Highest Academic Average Award}{\uscl}{2004}
}

%% Classes

\newcommand{\classes}{
  \class{\itology*}{2012---2013}{Open Source 101}{Open Source Software}
  \class{\itology}{2012---2013}{Version Control 101}{git, github}
  \class{\itology}{2012---2013}{Command Line 101}{Linux, Bash}
  \class{\usc*}{Fall 2011}{CSCE 211}{Digital Logic Design}
  \class{\usc}{Summer II 2008}{CSCE 102}{HTML/CSS/JavaScript}
  \class{\usc}{Spring 2008}{CSCE 145 Lab}{Algorithmic Design}
  \class{\usc}{Fall 2007}{CSCE 145 Lab}{Algorithmic Design}
  \class{\uscl*}{Spring 2007}{CSCE 204}{Program Design and Development}
  \class{\uscl}{Spring 2007}{Math 241 \& Math 242}{Maple}}

%% Service

\newcommand{\servicelist}{
  \service{Mentor}%
          {Groupon internship program}%
          {2014}
  \service{Book Reviewer}%
          {\href{https://www.packtpub.com/big-data-and-business-intelligence/practical-data-analysis}{Practical Data Analysis}, Packt Publishing}%
          {2013}
  \service{Webmaster}%
          {\href{https://cvl.cse.sc.edu/wvm2013/}{Winter Vision Meetings}}%
          {2013}
  \service{Webmaster}%
          {\href{https://cvl.cse.sc.edu/wacv2013/}{Workshop on the Applications of Computer Vision}}%
          {2013}
  \service{Judge}%
          {Discovery Day --- Undergraduate Research Presentations}%
          {2012}
  \service{Reviewer}%
          {Pattern Recognition Letters}%
          {2011---2012}
  \service{Reviewer}%
          {IEEE Transactions on Pattern Analysis and Machine Intelligence}%
          {2012}
  %% \service{Fellow}{NSF GK-12 Program}
  %% \service{Member}{Institute of Electrical and Electronics Engineers (IEEE)}
  \service{SysAdmin}{Computer Vision Lab}{2009---2012}
}

%% Footer

\newcommand{\activities}{open source software, GIS visualization,
  Linux, \href{https://soundcloud.com/malloc47}{music composition}}

\newcommand{\footerCV}{
  \null\vfill
  \footnotesize{
    Online: \href{http://cv.malloc47.com}{cv.malloc47.com} \hfill
    Résumé: \href{http://resume.malloc47.com}{resume.malloc47.com} \hfill
    Source: \href{https://github.com/malloc47/cv/tree/master}{github.com/malloc47/cv/}}}

\newcommand{\footerResume}{
  \null\vfill
  \footnotesize{
    Online:  \href{http://resume.malloc47.com}{resume.malloc47.com} \hfill
    Full CV: \href{http://cv.malloc47.com}{cv.malloc47.com} \hfill
    Source:  \href{https://github.com/malloc47/cv/tree/master}{github.com/malloc47/cv/}}}
