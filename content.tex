%% Intro

\newcommand{\biographical}{
  \begin{multicols}{2}
    \setlength{\parskip}{0pt}
    \section{Biographical}

    \begin{tabularx}{\linewidth}{@{}l X@{}}
      \textsc{Work Address}& \small{3000 K St NW \#350} \\
      & \small{Washington, DC 20007} \\
      \textsc{Phone}       & \href{tel:847-261-4747}{847-261-4747} \\
      \textsc{email}       & \href{mailto:jarrell.waggoner@gmail.com}{jarrell.waggoner@gmail.com} \\
    \end{tabularx}

    \vfill
    \columnbreak

    \section{Online}
    \begin{tabularx}{\linewidth}{@{}l X@{}}
      \textsc{Website}     & \href{http://www.malloc47.com}{www.malloc47.com} \\
      \textsc{Twitter}     & \href{https://twitter.com/malloc47}{@malloc47} \\
      \textsc{github}      & \href{http://www.github.com/malloc47}{github.com/malloc47}\\
      \textsc{LinkedIn}    & \href{http://www.linkedin.com/in/malloc47}{linkedin.com/in/malloc47} \\
    \end{tabularx}
  \end{multicols}
}

\newcommand{\interests}{computer vision, image processing, artificial
  intelligence, pattern recognition \& machine learning, data science,
  data engineering, functional programming, web development, GIS,
  Clojure}

\newcommand{\interestsLong}{computer vision, segmentation, document
  image analysis, event recognition, image processing, artificial
  intelligence, pattern recognition \& machine learning, data science,
  functional programming, GIS, Clojure}

%% Education

\newcommand{\phdDegree}{
  \degree{Aug. 2013}%
  {Ph.D.}%
  {Doctor}%
  {of}%
  {Computer Science \& Engineering}%
  {University of South Carolina}%
  {Dr. Song \textsc{Wang}}%
  {Multi-Label Segmentation Propagation for Materials Science Images
    Incorporating Topology and Interactivity}%
  {}}

\newcommand{\meDegree}{
  \degree{May 2009}%
  {M.E.}%
  {Master of Engineering}%
  {in}%
  {Computer Science}%
  {University of South Carolina}%
  {}%
  {}%
  {\small\emph{magna cum laude}}}

%% Experience

\newcommand{\rally}{
  \experience{2017--- \\ Present}%
  {Architect}%
  {at}%
  {\href{https://www.rallyhealth.com}{Rally Health, Inc.}}%
  {Seasoned engineering leader responsible for overseeing technical
    design and engineering decisions across Rally's entire data
    organization of 50+ engineers, analysts, and data scientists.

    \begin{exlist}
    \item Architected complete rewrite of the data platform for the
      whole company, moving from a fixed \skill{Cloudera} cluster to a
      self-service platform using \skill{Databricks} and
      \skill{Redshift} fed by \spark ETLs written in \scala and
      scheduled with \skill{Airflow} atop \skill{Kubernetes}. Built
      consensus on new architecture and delivered working system
      within a year while growing the team from only myself to an
      independent platform team of 10 engineers.

    \item Member of the Rally Engineering Technical Staff, responsible
      for making cross-cutting engineering decisions, evaluating
      potential acquisitions, signing off on major company-wide
      architectural changes, and organizing technical interest groups.

    \item Heavily involved in defining team structure and hiring,
      conducting over 100 interviews for IC and management roles to
      scale the data organization from 4 engineers to over 50
      engineers.

    \item Fostered good design practices across the data organization
      by auditing design documents, leading architecture meetings for
      multiple teams to grow design aptitude, and spearheading the
      standardization of best-practices on the team including code
      review, documented on-call workflow, CI/CD , monitoring/alerting,
      idempotent ETL workflows, and reproducible EDW layout.

    \item Coordinated with over 16 internal and external teams across
      an extensive range of projects including productionalized ML
      workflows, frontend/mobile event tracking, real time data
      processing, data anonymization, security/compliance/privacy
      requirements, data ingestion APIs, data quality validation, and
      self-service internal product analytics.

    \end{exlist}}}

\newcommand{\drw}{
  \experience{2016---2017}%
  {Software Engineer}%
  {at}%
  {\href{http://www.drw.com}{DRW Holdings, LLC}}%
  {Member of the Trading Infrastructure team, developing the internal
    platform used by every trading desk at DRW. Built greenfield
    high-performance service-oriented systems using \clojure and \java
    while maintaining legacy applications in \ruby and \csh among a
    catalog of over 50 microservices.

    \begin{exlist}
    \item Contributed to a \ruby-based reconciliation tool used to
      balance cash flows for high-volume trading

    \item Extended a research workflow tool used for computing the
      value and settle price of options, futures, equities, and other
      financial instruments, written in \clojure.

    \item Developed and extended multiple UI frontends for internal
      tools using \skill{React} and \skill{Reagent}.
  \end{exlist}}}

\newcommand{\groupon}{
  \experience{2013---2016}%
  {Senior Software Engineer}%
  {at}%
  {\href{http://www.groupon.com}{Groupon, Inc.}}%
  {Contributed to three engineering teams: The Flux team building Data
    Science pipelines, the Project Genesis strike team integrating
    scraped web data into \salesforce, and served as Tech Lead of the
    Supply Intelligence team creating internal sales tools to optimize
    Groupon's supply funnel.

    \begin{exlist}
    \item Built a \postgres-backed high-performance caching and write
      management system in \clojure around the \salesforce API that
      hits 10K req/min.

    \item Managed critical business automation of the sales lead
      assignment process that previously required an estimated 80
      sales managers to conduct manually; led the effort to
      rearchitect this legacy system from an ad-hoc job scheduling
      platform written in \ruby and \bash to a multi-staged \hadoop
      pipeline written in \clojure to handle over 6M accounts.

  %% \item Operationalized \python within the organization by spinning
  %%   up an internal PyPI server; introducing a pex-driven
  %%   single-artifact deployment process; creating an interest group,
  %%   mailing list, and internal wiki; and standardized configuration,
  %%   logging, and resource management.

    \item Coordinated with product and business teams to ETL 250K
      leads in \salesforce from scraped web data.

    \item Built out an ETL management and machine learning platform
      using \python, \clojure, \hive, and \spark to run
      mission-critical Decision Tree Learning models to predict
      customer attrition, lifetime customer value, and merchant value.

    \item Mentored interns and junior developers, established best
      practices, and led multiple major technical initiatives on a
      team of 5 developers.
    \end{exlist}}}

\newcommand{\terrastride}{
  \experience{2012---2014}%
  {Technical Engineer}%
  {at}%
  {\href{http://www.terrastride.com/}{TerraStride, Inc.}}%
  {Software developer in an agile startup environment creating the
    \href{http://www.huntstand.com}{huntstand.com} web application.
    Written using \python, \django, and \backbone; deployed to
    \skill{AWS}.  Responsible for curating full technology stack and
    coordinating with $5$ developers.}}

\newcommand{\cvl}{
  \experience{2011---2013}%
  {Research Assistant}%
  {at}%
  {USC \href{http://cvl.cse.sc.edu/}{Computer Vision Lab}}%
  {Dissertation research on computer vision models and algorithms for
    materials science image segmentation in \python, \numpy, \scipy,
    \opencv, and \matlab.  Created a web interface using \django, \js,
    and \jquery. Conducted large-scale analysis using a 98-core
    high-performance computing system.}}

\newcommand{\palmettocomputerlabs}{
  \experience{2011---2013}%
  {Project Manager}%
  {at}%
  {\href{http://palmettocomputerlabs.com/}{Palmetto Computer Labs}}%
  {Created and taught workshops on \git, the \linux command line,
    \android development, and open source software for hundreds of
    students, developers, and government officials at
    \institution{\href{http://it-ology.org/}{IT-oLogy}}.  Managed the
    \institution{\href{http://open-it-lab.com/}{Open IT Lab}} and
    associated projects. Assisted in planning
    \institution{\href{http://posscon.org/}{POSSCON}}.}}

\newcommand{\elasticvision}{
  \experience{2011}%
  {Contractor}%
  {for}%
  {Elastic Vision Consulting}%
  {Built a parser and generator for \skill{XML} medical records
    formats (CCR and CCD) in a \java web application.  Written using
    \skill{JDOM}, \skill{Xerces}, and \skill{Hibernate}, and run on an
    \skill{Axis2+Jetty6} driven server.}}

\newcommand{\darpa}{
  \experience{2010---2011}%
  {Research Assistant}%
  {for the}%
  {DARPA
    \href{https://en.wikipedia.org/wiki/Mind\%27s_Eye_(US_military)}{Mind's
      Eye Program} }%
  {Researched video event recognition for the DARPA Mind's Eye
    program.  Collaborated with $10$ students and faculty members
    across three institutions.  Developed algorithms in \scheme,
    \bash, \matlab, and \c to process a corpus of 3480 videos
    extracted into over 1.5 million frames. Distributed processing
    over $7$ HPC machines.
    \href{http://0xab.com/research/video-events.html}{0xab.com/research/video-events.html}
    ,
    \href{https://www.github.com/malloc47/video-in-sentences-out}{github.com/malloc47/video-in-sentences-out}}}

\newcommand{\neh}{
  \experience{2009---2010}%
  {NEH Fellow}%
  {at the}%
  {\href{https://sc.edu/about/centers/digital_humanities/index.php}{USC Center for Digital Humanities}
  (Sapheos/\href{http://sc.edu/about/centers/digital_humanities/projects/paragon.php}{Paragon}
  Project)}%
  {Developed the prototype for a \emph{digital collation} application
    to identify sub-textual inconsistencies among multiple copies of
    \emph{The Faerie Queene} by \textsc{Edmund Spenser}.  Created in
    \matlab using \skill{VLFeat} and \opencv to process tens of
    thousands of book page images.
    \href{https://www.github.com/malloc47/digital-collation}{github.com/malloc47/digital-collation}}}

\newcommand{\cse}{
  \experience{2007---2011}%
  {Teaching Assistant}%
  {for}%
  {\href{https://www.cse.sc.edu/}{USC Department of Computer Science
      and Engineering}}%
  {Taught classes in software development, web development, and
    computer engineering, covering \java, \js, \html, and
    \skill{Visual Basic}.  Created syllabi and course objectives,
    developed and graded projects and assignments, supervised labs,
    and tutored students.}}

\newcommand{\internship}{
  \experience{2005}%
  {Intern---Technical Writer}%
  {at}%
  {JAARS, Inc.}%
  {Created documentation and integrated context-sensitive online help
    system for speech and linguistic software written in C++ and
    Visual Basic.}}

\newcommand{\volunteering}{
  \experience{2001---2002}%
  {Volunteer Software Developer}%
  {at}%
  {JAARS, Inc.}%
  {Spearheaded the conversion from \skill{Visual Basic 4} to
    \skill{Visual Basic 6} for the linguistic reference tool
    \href{http://www.sil.org/computing/ipahelp/ipaprvw2.htm}{IPA
      Help}.}}

\newcommand{\afosr}{
  \academic{2011---2013}%
  {Research Assistant funded by \textsc{AFOSR}}%
  {Materials Volume Segmentation}%
  {Developed segmentation methods for materials image volumes in
  \emph{Python+NumPy/SciPy} and \emph{MATLAB} at the \textsc{Computer
    Vision Lab} at \textsc{USC}. Managed the lab computer network and
  organized weekly lab meetings.  Created GUI interface using
  wxWidgets for assisted segmentation, and conducted large-scale
  evaluations on multiple datasets for metallic and biological
  materials.}}

\newcommand{\darpaRA}{
  \academic{2010---2011}%
  {Research Assistant funded by \textsc{DARPA}}%
  {Video Event Recognition}%
  {Explored segmentation methods for video event recognition. Attended
    P.I. meetings in San Diego (2010) and Colorado (2011). Developed
    algorithms in \emph{Scheme} to process a corpus of thousands of
    videos extracted into over 3 million frames using a
    high-performance computing cluster.}}

\newcommand{\nehfellow}{
  \academic{2009---2010}%
  {NEH Fellow at the \textsc{Center for Digital Humanities}}%
  {Digital Collation}%
  {Created a \textsc{digital collation} application to handle
    automatic differencing of sub-textual inconsistencies among
    multiple copies of \emph{The Faerie Queene} by \textsc{Edmund
      Spenser} in \emph{MATLAB} to process tens of thousands of book
    page images.}}

\newcommand{\crayton}{
  \academic{2008---2009}%
  {GK-12 Fellow at \textsc{Crayton Middle School}}%
  {8\textsuperscript{th} Grade Science}%
  {Served in Crayton Middle School, coordinating with the classroom
    instructor to enhance the science curriculum and activities in an
    8\textsuperscript{th} grade science classroom. Subsequently
    coordinated and taught at the \textsc{GK-12 Institute for
      Teachers}, presenting the activities developed and delivered in
    the classroom.}}

\newcommand{\cseTA}{
  \academic{2007---2008, 2011}%
  {Graduate Teaching Assistant at \textsc{USC}}%
  {Web Development}%
  {Supervised CSCE~145 labs, covering software development with
    \textsc{Java}, and taught CSCE~102, covering \textsc{Javascript},
    \textsc{HTML}, and \textsc{CSS}. Taught~CSCE~211 covering digital
    logic design.}}

\newcommand{\usclTA}{
  \academic{Spring 2007}%
  {Instructor for \textsc{CSCE 204} at \textsc{USCL}}%
  {Introductory  Programming}%
  {Hired as special faculty. Taught introductory Visual Basic for
    majors and non-majors. Selected textbooks, developed all course
    material, graded all assignments. Worked with Dr. Noni
    M. Bohonak}}

\newcommand{\campInstructor}{
  \academic{Fall 2006}%
  {Camp Instructor for \textsc{USCL Arts and Sciences Adventure Camp}}%
  {5\textsuperscript{th}-8\textsuperscript{th} Grade Students}%
  {Worked in collaboration with Dr. Dwayne Brown. One of two
    instructors teaching Math and Computer Science to grade school
    students.}}

\newcommand{\tutor}{
  \academic{2003---2007}%
  {Professional Tutor at \textsc{USCL Academic Success Center}}%
  {High School and College Students}%
  {Student and graduate tutor for college-level Mathematics, Computer
    Science, Physics, and English classes.}}

%% Skills

\newcommand{\skillsList}{
  \skills{\bash}{\threeskill}
  \skills{\clojure}{\threeskill}
  \skills{\git}{\threeskill}
  \skills{GNU/\linux}{\threeskill}
  \skills{\footnotesize{\hadoop ecosystem}}{\threeskill}
  \skills{\haskell}{\oneskill}
  \skills{\java}{\threeskill}
  \skills{\js}{\twoskill}
  \skills{\LaTeX}{\twoskill}
  \skills{\skill{Nix/NixOS}}{\oneskill}
  \skills{\postgres}{\threeskill}
  \skills{\python}{\threeskill}
  \skills{\scala}{\twoskill}
  \skills{\scheme}{\threeskill}
  \skills{\spark}{\threeskill}
}

\newcommand{\skillsListLong}{
  \skills{\bash}{\threeskill}
  \skills{\ccpp}{\twoskill}
  \skills{\clojure}{\threeskill}
  \skills{\django}{\twoskill}
  \skills{Emacs Lisp}{\twoskill}
  \skills{\git}{\threeskill}
  \skills{GNU/\linux}{\threeskill}
  \skills{\hadoop}{\threeskill}
  \skills{\haskell}{\oneskill}
  \skills{\java}{\threeskill}
  \skills{\js}{\twoskill}
  \skills{\LaTeX}{\twoskill}
  \skills{\matlab}{\twoskill}
  \skills{\numpy/\scipy}{\threeskill}
  \skills{\opencv}{\threeskill}
  \skills{\postgres}{\threeskill}
  \skills{\python}{\threeskill}
  \skills{\scala}{\threeskill}
  \skills{\scheme}{\threeskill}
  \skills{\spark}{\threeskill}
}

\newcommand{\skillsLegend}{
  \begin{footnotesize}
    \oneskill Small-scale or personal projects \hfill
    \twoskill Used in production  \hfill
    \threeskill Used in large-scale production systems
  \end{footnotesize}}

% graveyard

%% \skills{Blender}{\twoskill}
%% \skills{English}{\threeskill}
%% \skills{LAMP Stack}{\fournotes}
%% \skills{LISP}{\onenote}
%% \skills{Learning}{\fournotes Machine}
%% \skills{MS Office}{\fivenotes}
%% \skills{Maple}{\twoskill}
%% \skills{Networking}{\threenotes}
%% \skills{Processing}{\fivenotes Image}
%% \skills{Sys. Admin.}{\threenotes}
%% \skills{Visual Basic}{\fivenotes}
%% \skills{Windows}{\fivenotes}
%% \skills{Wordpress}{\fournotes}
%% \skills{\hive}{\oneskill}
%% \skills{\html}{\threeskill}
%% \skills{\jquery}{\twoskill}
%% \skills{\php}{\oneskill}


%% Projects

\newcommand{\matsciseg}{
  \proj{matsciseg}%
  {Framework for propagated 3D volume segmentation, used in my
  dissertation work.  Algorithms created in \python and \cpp and
  exposed as a web API using \django. Includes a web application that
  consumes the API created in \js, and \jquery.}%
  {github.com/malloc47/matsciseg}}

\newcommand{\nonpartisanme}{
  \proj{\href{http://nonpartisan.me}{nonpartisan.me}}%
  {Google Chrome extension that filters social media websites for
  political keywords.  Available in the
  \href{https://chrome.google.com/webstore/detail/nonpartisanme/ninebcppidndhampaggnjbijpacoadgg}{Chrome
    Web Store}.  Featured in the
  \href{http://www.charlestoncitypaper.com/charleston/sick-of-politics-on-facebook-try-this-browser-tool/Content?oid=4153447}{Charleston
    City Paper}.}%
  {github.com/malloc47/nonpartisan.me}}

\newcommand{\termdo}{
  \proj{term-do}%
  {An interactive terminal prompt that displays potential command
    completions as you type.  A hybrid of gnome-do and Emacs's
    ido-mode.  Works on many tested VT100 terminal types; built
    in~\skill{C++}.  Includes client/server architecture implemented
    with boost.interprocess and full-featured plugin system.
    Available in the
    \href{https://aur.archlinux.org/packages/term-do-git/}{Arch Linux
      AUR}.}%
  {github.com/malloc47/term-do}}

\newcommand{\ratiocontour}{
  \proj{Ratio Contour}%
  {Maintainer and contributor for the Ratio Contour project, a salient
  object detection and segmentation method used for computer vision
  applications.  Developed in \skill{C} and \skill{MATLAB}.}%
  {github.com/malloc47/ratio-contour}}

\newcommand{\digitalcollation}{
  \proj{Digital Collation}%
  {Research project to ``collate'' high-resolution documents by using
    image registration, accomplished using the SIFT feature detector
    and a thin plate spline warping technique, written in MATLAB.}%
  {github.com/malloc47/digital-collation}}

\newcommand{\pmldap}{
  \proj{PMLDAP}%
  {\skill{Linux} user management tool for Linux clusters.  Created as
    a simplified replacement for LDAP.  Capable of bootstrapping new
    systems, synchronizing users and configuration files, and running
    distributed commands.  Written in \skill{Bash}.}%
  {github.com/malloc47/pmldap}}

\newcommand{\matscicut}{
  \proj{matscicut}%
  {An energy minimization framework for segmenting 3D materials
    volumes. Prototype of dissertation work, created in C++ using
    OpenCV libraries, with assorted MATLAB helper utilities.}%
  {github.com/malloc47/matscicut}}

\newcommand{\githq}{
  \proj{git-hq}%
  {A remote management system for git, created in Python.}%
  {github.com/malloc47/git-hq}}

\newcommand{\sinaweibo}{
  \proj{Sina Weibo Mobile Client}%
  {Created a \skill{J2ME}-based prototype mobile client for the
    popular Chinese \institution{Sina} microblogging service, similar
    to \institution{Twitter}.  Targeted at limited-functionality CLDC
    phones and uses a custom \skill{Java} wrapper for the
    \institution{Sina} API.  Employs symmetric-key encryption for
    personal data.}%
  {bd.weibo.10086.cn/2012/downloads\_kjav}}

\newcommand{\befungepy}{
  \proj{befunge.py}
  {Complete \href{https://en.wikipedia.org/wiki/Befunge}{Befunge}
    interpreter written in \python.  Implements the Befunge 93
    specification, and is one of the closest Python equivalents to the
    \c reference implementation.}
  {github.com/malloc47/befunge.py}}

%% References

\newcommand{\referenceslist}{
  \nocite{derrick:16}
  \nocite{waggoner:15}
  \nocite{zhou:14}
  \nocite{waggoner:phd}
  \nocite{waggoner:14}
  \nocite{waggoner:13a}
  \nocite{waggoner:13c}
  \nocite{waggoner:11}
  \nocite{wang:11}
  \nocite{temlyakov:10}
  \nocite{zhang:10}
  \nocite{waggoner:12}
  \nocite{barbu:12}
  \nocite{barbu:12b}
  \nocite{zhang:12}
  \nocite{temlyakov:13}
  \nocite{salvi:13a}
  \nocite{salvi:13b}
}

\newcommand{\referencesShort}{
  \nocite{derrick:16}
  \nocite{waggoner:15}
  \nocite{waggoner:phd}
  \nocite{waggoner:14}
  \nocite{waggoner:13a}
  \nocite{barbu:12}
  \nocite{waggoner:11}
}

\newcommand{\presentations}{
  \nocite{uhg:19}
  \nocite{posscon:15}
  \nocite{ato:13}
  \nocite{gsd:13}
  \nocite{posscon:13}
  \nocite{uscl:13}
  \nocite{spie:13}
  \nocite{appathon:12}
  \nocite{uscsymposium:12}
  \nocite{scmta:12}
  \nocite{sctcs:12}
  \nocite{dhhpc:12}
  \nocite{spie:12}
  \nocite{scgmis:12}
  \nocite{darpa:11}
}

\newcommand{\presentationsShort}{
  \nocite{uhg:19}
  \nocite{posscon:15}
  \nocite{ato:13}
  \nocite{posscon:13}
}

%% Guest Lectures

%% \begin{enumerate}
%% \renewcommand{\labelenumi}{[G\arabic{enumi}] }
%% %% \item \emph{Combining Global Labeling and Local Relabeling for Metallic Image Segmentation}. Graduate Student Day Competition, Second Place. April 8, 2011.
%% %% \item \emph{Image Registration for Digital Collation}. Graduate Student Day Competition, Honorable Mention. April 2, 2010.
%% \item \emph{Building Chrome Extensions}.  In CSCE 242.  Guest lecture for Dr. José M. Vidal.  November~30, 2012.
%% \item \emph{Modeling in Blender}.  In CSCE 552.  Guest lecture for Dr. Jijun Tang.  February~28, 2011.
%% \item \emph{Aspect-Oriented Programming}. In CSCE 531. Guest lecture for Dr. Marco Valtorta. March 19, 2008.
%% \item \emph{Math 241}. Vector Calculus. Guest lecture for Dr. Dwayne Brown. April~23---26, 2007.
%% \item \emph{Math 242}. Differential Equations. Guest lecture for Dr. Dwayne Brown. April~23---26, 2007.
%% \end{enumerate}

%% Travel

%\begin{enumerate}
%\renewcommand{\labelenumi}{[T\arabic{enumi}] }
%\item \emph{Mind's Eye PI Meeting}. DARPA Project P.I.~meeting. Denver, CO. January 20---21, 2011.
%\item \emph{Tomography and its Applications to Materials Science and Non-Destructive Evaluation}. Organizd by M. De Graef, L. Drummy, J. Simmons, M. Comer, C. Bouman, and J. Knopp. Tech\^{}Edge, Dayton, Ohio. December 13---15, 2010.
%\item Visiting scholar. In collaboration with J. M. Siskind. Purdue University, West Lafayette, IN. December 6---22, 2010 \& January 5---16, 2011.
%\item \emph{Mind's Eye Kickoff Meeting}. DARPA Project P.I.~meeting. San Diego, CA. September 23---24, 2010.
%\end{enumerate}


%% Honors/Awards

\newcommand{\awardlist}{
  \award{First Place}{Gamecock Computing Research Symposium Poster Session}{2012}
  \award{First Place}{Graduate Student Day Presentation}{2012}
  \award{Second Place}{Graduate Student Day Presentation}{2011}
  \award{Honorable Mention}{Graduate Student Day Presentation}{2010}
  \award{Inductee}{Upsilon Pi Epsilon}{2009}
  \award{Recipient}{Clara P. Hammond Award}{2004}
  \award{Recipient}{Science and Mathematics Award}{2004}
  \award{Recipient}{Highest Academic Average Award}{2004}
}

%% Classes

\newcommand{\itologyOS}{
  \class{2012---2013}{Open Source 101}{Open Source Software}}
\newcommand{\itologyVC}{
  \class{2012---2013}{Version Control 101}{git, github}}
\newcommand{\itologyCL}{
  \class{2012---2013}{Command Line 101}{Linux, BASH}}
\newcommand{\uscCCI}{
  \class{Fall 2011}{CSCE 211}{Digital Logic Design}}
\newcommand{\uscCII}{
  \class{Summer II 2008}{CSCE 102}{HTML/CSS/JavaScript}}
\newcommand{\uscCXLVSpring}{
  \class{Spring 2008}{CSCE 145 Lab}{Java}}
\newcommand{\uscCXLVFall}{
  \class{Fall 2007}{CSCE 145 Lab}{Java}}
\newcommand{\usclCCIV}{
  \class{Spring 2007}{CSCE 204}{Visual Basic}}
\newcommand{\usclMath}{
  \class{Spring 2007}{Math 241 \& Math 242}{Maple}}

%% Service

\newcommand{\servicelist}{
  \service{Mentor}%
          {Groupon internship program}%
          {2014}
  \service{Book Reviewer}%
          {\href{https://www.packtpub.com/big-data-and-business-intelligence/practical-data-analysis}{Practical Data Analysis}, Packt Publishing}%
          {2013}
  \service{Webmaster}%
          {\href{http://cvl.cse.sc.edu/wvm2013/}{Winter Vision Meetings}}%
          {2013}
  \service{Webmaster}%
          {\href{http://cvl.cse.sc.edu/wacv2013/}{Workshop on the Applications of Computer Vision}}%
          {2013}
  \service{Judge}%
          {Discovery Day --- Undergraduate Research Presentations}%
          {2012}
  \service{Reviewer}%
          {Pattern Recognition Letters}%
          {2011---2012}
  \service{Reviewer}%
          {IEEE Transactions on Pattern Analysis and Machine Intelligence}%
          {2012}
  %% \service{Fellow}{NSF GK-12 Program}
  %% \service{Member}{Institute of Electrical and Electronics Engineers (IEEE)}
  \service{SysAdmin}{Computer Vision Lab}{2009---2012}
}

%% Footer

\newcommand{\activities}{open source software, GIS visualization,
  Linux, \href{https://soundcloud.com/malloc47}{music composition}}

\newcommand{\footer}{
  \null\vfill
  \footnotesize{
    Online:  \href{http://cv.malloc47.com}{cv.malloc47.com} \hfill
    Résumé: \href{http://resume.malloc47.com}{resume.malloc47.com} \hfill
    Source:  \href{https://github.com/malloc47/cv/tree/master}{github.com/malloc47/cv/}}}
