\usepackage{bibunits}
\usepackage{changepage}
\usepackage{enumitem}
\usepackage{mdframed}
\usepackage{multicol}
\usepackage{multirow}
\usepackage{tabularx}
\usepackage{textpos}
\usepackage{titlesec}
\usepackage[usenames,dvipsnames]{xcolor}
\usepackage{xstring}
\usepackage{xunicode,xltxtra,url,parskip}
\usepackage{xparse}


%Setup hyperref package, and colours for links
\definecolor{linkcolour}{rgb}{0,0.2,0.6}
\hypersetup{colorlinks,breaklinks,urlcolor=linkcolour, linkcolor=linkcolour}

% Renders bracketed links, where the full \linkf form will show the
% shortened URL (so it appears in printouts).
\newcommand{\link}[2]{{[}\href{#2}{#1}{]}}
\newcommand{\linkf}[2]{%
  {[}#1: \href{#2}{%
    %% \StrDel[1]{\StrDel[1]{\StrDel[1]{#2}{https://}}{http://}}{www.}%
    \StrDel[1]{#2}{https://}[\result]%
    \StrDel[1]{\result}{http://}[\result]%
    \StrDel[1]{\result}{www.}%
  }{]}%
}%

%Color
\definecolor{lightg}{HTML}{999999}
\definecolor{medg}{HTML}{666666}
\definecolor{darkg}{HTML}{444444}

%Experience section
\newenvironment{exlist}
  {\begin{itemize}[
      leftmargin=1.5em,
      topsep=-0.5em,
      itemsep=0pt,
      parsep=0.5ex,
      partopsep=0pt,
    ]
      \renewcommand\labelitemi{---}}
  {\end{itemize}}

\newmdenv[
  topline=false,
  bottomline=false,
  rightline=false,
  leftmargin=-1ex,
  rightmargin=0,
  skipabove=0,
  skipbelow=0,
  innerrightmargin=0,
  innerleftmargin=1ex,
  innertopmargin=0,
  innerbottommargin=0
]{sideblock}

\DeclareDocumentCommand \experience {m m m m o o o o +m}{%
\begin{normalsize}
  \begin{tabularx}{\textwidth}{@{}l|Xll@{}}
    \begin{large}\textsc{#1}\end{large} & #2 & \textcolor{darkg}{#3} & \textcolor{darkg}{#4} \\
    \IfNoValueF{#5}{\IfNoValueF{#8}{#8} & #5 & \textcolor{darkg}{#6} & \textcolor{darkg}{#7}}
  \end{tabularx}
\end{normalsize}
\begin{small}#9\end{small}

\vspace{\baselineskip}}

% Bullets
\definecolor{noteone}{HTML}{999999}
\definecolor{notetwo}{HTML}{848484}
\definecolor{notethree}{HTML}{424242}
\definecolor{notefour}{HTML}{212121}
\definecolor{notefive}{HTML}{000000}

\newcommand{\fivenotes}{%
	\textcolor{noteone}{\symbol{"2022}}
	\textcolor{notetwo}{\symbol{"2022}}
	\textcolor{notethree}{\symbol{"2022}}
	\textcolor{notefour}{\symbol{"2022}}
	\textcolor{notefive}{\symbol{"2022}}
}
\newcommand{\fournotes}{%
	\textcolor{noteone}{\symbol{"2022}}
	\textcolor{notetwo}{\symbol{"2022}}
	\textcolor{notethree}{\symbol{"2022}}
	\textcolor{notefour}{\symbol{"2022}}
	\textcolor{white}{\symbol{"2022}}
}
\newcommand{\threenotes}{%
	\textcolor{noteone}{\symbol{"2022}}
	\textcolor{notetwo}{\symbol{"2022}}
	\textcolor{notethree}{\symbol{"2022}}
	\textcolor{white}{\symbol{"2022}}
	\textcolor{white}{\symbol{"2022}}
}
\newcommand{\twonotes}{%
	\textcolor{noteone}{\symbol{"2022}}
	\textcolor{notetwo}{\symbol{"2022}}
	\textcolor{white}{\symbol{"2022}}
	\textcolor{white}{\symbol{"2022}}
	\textcolor{white}{\symbol{"2022}}
}
\newcommand{\onenote}{%
	\textcolor{noteone}{\symbol{"2022}}
	\textcolor{white}{\symbol{"2022}}
	\textcolor{white}{\symbol{"2022}}
	\textcolor{white}{\symbol{"2022}}
	\textcolor{white}{\symbol{"2022}}
}

\newcommand{\oneskill}{%
  \textcolor{white}{\symbol{"2022}}
  \textcolor{white}{\symbol{"2022}}
  \textcolor{notefive}{\symbol{"2022}}
}

\newcommand{\twoskill}{%
  \textcolor{white}{\symbol{"2022}}
  \textcolor{notethree}{\symbol{"2022}}
  \textcolor{notefive}{\symbol{"2022}}
}

\newcommand{\threeskill}{%
  \textcolor{noteone}{\symbol{"2022}}
  \textcolor{notethree}{\symbol{"2022}}
  \textcolor{notefive}{\symbol{"2022}}
}

%FONTS
% \defaultfontfeatures{Mapping=tex-text}
% \setmainfont[SmallCapsFont = Fontin SmallCaps]{Fontin}

\setromanfont [Ligatures={Common}, BoldFont={Linux Libertine O Bold}, ItalicFont={Linux Libertine O Italic}]{Linux Libertine O}
\setsansfont [Ligatures={Common}, BoldFont={GeosansLight}, ItalicFont={GeosansLight}]{GeosansLight}
\setmonofont{GeosansLight}

\font\lighttext=''LibreBaskerville-Regular:color=787878'' at 10pt
\font\lighttextweb=''LibreBaskerville-Regular:color=FF1493'' at 10pt

\newfontfamily\sans{DejaVu Sans}

%CV Sections inspired by:
%http://stefano.italians.nl/archives/26
\titleformat{\section}{\Large\scshape\raggedright\sffamily}{}{0em}{}[\titlerule]
%% \titlespacing{\section}{0pt}{-2pt}{0pt}

%-------------WATERMARK TEST [**not part of a CV**]---------------
\TPGrid[30mm,30mm]{30}{60}
%\setlength{\TPHorizModule}{30mm}
%\setlength{\TPVertModule}{\TPHorizModule}
%\textblockorigin{2mm}{0.65\paperheight}
\setlength{\parindent}{0pt}

\newcommand{\skill}{\textbf}
\newcommand{\skills}[2]{
  \item #2 #1
}

\newcommand{\institution}{\textsc}

% \def\bullet{\textcolor{medg}{\symbol{"00BB}}}
\def\div{\,\textbar{}\,}

\usepackage{xspace}

% language macros

\newcommand{\lang}[2]{\expandafter\def\csname #1\endcsname{\skill{#2}\xspace}}

\lang{airflow}{Airflow}
\lang{android}{Android}
\lang{backbone}{Backbone.js}
\lang{bash}{Bash}
\lang{cascalog}{Cascalog}
\lang{ccpp}{C/C++}
\lang{clojurescript}{ClojureScript}
\lang{clojure}{Clojure}
\lang{cpp}{C++}
\lang{csh}{C\#}
\lang{c}{C}
\lang{django}{Django}
\lang{git}{git}
\lang{hadoop}{Hadoop}
\lang{haskell}{Haskell}
\lang{hive}{Hive}
\lang{html}{HTML/CSS}
\lang{java}{Java}
\lang{jquery}{jQuery}
\lang{js}{JavaScript}
\lang{linux}{Linux}
\lang{matlab}{MATLAB}
\lang{numpy}{NumPy}
\lang{opencv}{OpenCV}
\lang{php}{PHP}
\lang{postgres}{PostgreSQL}
\lang{python}{Python}
\lang{react}{React}
\lang{reagent}{Reagent}
\lang{reframe}{re-frame}
\lang{ruby}{Ruby}
\lang{salesforce}{Salesforce}
\lang{scala}{Scala}
\lang{scheme}{Scheme}
\lang{scipy}{SciPy}
\lang{spark}{Spark}
\lang{sql}{SQL}
\lang{teradata}{Teradata}
\lang{visb}{Visual Basic}
