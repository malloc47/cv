\documentclass[10pt]{article}

\usepackage{marvosym}
\usepackage{fontspec}
\usepackage{xunicode,xltxtra,url,parskip}
\defaultfontfeatures{Scale=MatchLowercase,Mapping=tex-text}
\RequirePackage{color,graphicx}
\usepackage[usenames,dvipsnames]{xcolor}
\usepackage[left=1in, right=1in, top=1in, bottom=1in]{geometry}
\usepackage{supertabular}
\usepackage{titlesec}
\usepackage{multicol}
\usepackage{multirow}
\usepackage{longtable}
\usepackage{xstring}
\usepackage{rotating}
\usepackage{ifthen}
\usepackage[xetex,
            unicode,
            pdfencoding=auto,
            pdfinfo={
              Title={malloc47/cv},
              Author={Jarrell Waggoner},
              Subject={Jarrell Waggoner CV},
              Keywords={computer vision, image processing, artificial intelligence, pattern recognition, machine learning, data science, functional programming, web development, GIS, Clojure},
              Producer={xelatex},
              Creator{xelatex}
            },
            ]{hyperref}
\usepackage[absolute]{textpos}
\usepackage{enumitem}
\usepackage{tabularx}

\makeatletter
\renewcommand*{\@biblabel}[1]{\hfill[C#1]}
\makeatother

\usepackage{bibunits}
\usepackage{changepage}
\usepackage{enumitem}
\usepackage{mdframed}
\usepackage{multicol}
\usepackage{multirow}
\usepackage{tabularx}
\usepackage{textpos}
\usepackage{titlesec}
\usepackage[usenames,dvipsnames]{xcolor}
\usepackage{xstring}
\usepackage{xunicode,xltxtra,url,parskip}
\usepackage{xparse}


%Setup hyperref package, and colours for links
\definecolor{linkcolour}{rgb}{0,0.2,0.6}
\hypersetup{colorlinks,breaklinks,urlcolor=linkcolour, linkcolor=linkcolour}

% Renders bracketed links, where the full \linkf form will show the
% shortened URL (so it appears in printouts).
\newcommand{\link}[2]{{[}\href{#2}{#1}{]}}
\newcommand{\linkf}[2]{%
  {[}#1: \href{#2}{%
    %% \StrDel[1]{\StrDel[1]{\StrDel[1]{#2}{https://}}{http://}}{www.}%
    \StrDel[1]{#2}{https://}[\result]%
    \StrDel[1]{\result}{http://}[\result]%
    \StrDel[1]{\result}{www.}%
  }{]}%
}%

%Color
\definecolor{lightg}{HTML}{999999}
\definecolor{medg}{HTML}{666666}
\definecolor{darkg}{HTML}{444444}

%Experience section
\newenvironment{exlist}
  {\begin{itemize}[
      leftmargin=1.5em,
      topsep=-0.5em,
      itemsep=0pt,
      parsep=0.5ex,
      partopsep=0pt,
    ]
      \renewcommand\labelitemi{---}}
  {\end{itemize}}

\newmdenv[
  topline=false,
  bottomline=false,
  rightline=false,
  leftmargin=-1ex,
  rightmargin=0,
  skipabove=0,
  skipbelow=0,
  innerrightmargin=0,
  innerleftmargin=1ex,
  innertopmargin=0,
  innerbottommargin=0
]{sideblock}

\DeclareDocumentCommand \experience {m m m m o o o o +m}{%
\begin{normalsize}
  \begin{tabularx}{\textwidth}{@{}l|Xll@{}}
    \begin{large}\textsc{#1}\end{large} & #2 & \textcolor{darkg}{#3} & \textcolor{darkg}{#4} \\
    \IfNoValueF{#5}{\IfNoValueF{#8}{#8} & #5 & \textcolor{darkg}{#6} & \textcolor{darkg}{#7}}
  \end{tabularx}
\end{normalsize}
\begin{small}#9\end{small}

\vspace{\baselineskip}}

% Bullets
\definecolor{noteone}{HTML}{999999}
\definecolor{notetwo}{HTML}{848484}
\definecolor{notethree}{HTML}{424242}
\definecolor{notefour}{HTML}{212121}
\definecolor{notefive}{HTML}{000000}

\newcommand{\fivenotes}{%
	\textcolor{noteone}{\symbol{"2022}}
	\textcolor{notetwo}{\symbol{"2022}}
	\textcolor{notethree}{\symbol{"2022}}
	\textcolor{notefour}{\symbol{"2022}}
	\textcolor{notefive}{\symbol{"2022}}
}
\newcommand{\fournotes}{%
	\textcolor{noteone}{\symbol{"2022}}
	\textcolor{notetwo}{\symbol{"2022}}
	\textcolor{notethree}{\symbol{"2022}}
	\textcolor{notefour}{\symbol{"2022}}
	\textcolor{white}{\symbol{"2022}}
}
\newcommand{\threenotes}{%
	\textcolor{noteone}{\symbol{"2022}}
	\textcolor{notetwo}{\symbol{"2022}}
	\textcolor{notethree}{\symbol{"2022}}
	\textcolor{white}{\symbol{"2022}}
	\textcolor{white}{\symbol{"2022}}
}
\newcommand{\twonotes}{%
	\textcolor{noteone}{\symbol{"2022}}
	\textcolor{notetwo}{\symbol{"2022}}
	\textcolor{white}{\symbol{"2022}}
	\textcolor{white}{\symbol{"2022}}
	\textcolor{white}{\symbol{"2022}}
}
\newcommand{\onenote}{%
	\textcolor{noteone}{\symbol{"2022}}
	\textcolor{white}{\symbol{"2022}}
	\textcolor{white}{\symbol{"2022}}
	\textcolor{white}{\symbol{"2022}}
	\textcolor{white}{\symbol{"2022}}
}

\newcommand{\oneskill}{%
  \textcolor{white}{\symbol{"2022}}
  \textcolor{white}{\symbol{"2022}}
  \textcolor{notefive}{\symbol{"2022}}
}

\newcommand{\twoskill}{%
  \textcolor{white}{\symbol{"2022}}
  \textcolor{notethree}{\symbol{"2022}}
  \textcolor{notefive}{\symbol{"2022}}
}

\newcommand{\threeskill}{%
  \textcolor{noteone}{\symbol{"2022}}
  \textcolor{notethree}{\symbol{"2022}}
  \textcolor{notefive}{\symbol{"2022}}
}

%FONTS
% \defaultfontfeatures{Mapping=tex-text}
% \setmainfont[SmallCapsFont = Fontin SmallCaps]{Fontin}

\setromanfont [Ligatures={Common}, BoldFont={Linux Libertine O Bold}, ItalicFont={Linux Libertine O Italic}]{Linux Libertine O}
\setsansfont [Ligatures={Common}, BoldFont={GeosansLight}, ItalicFont={GeosansLight}]{GeosansLight}
\setmonofont{GeosansLight}

\font\lighttext=''LibreBaskerville-Regular:color=787878'' at 10pt
\font\lighttextweb=''LibreBaskerville-Regular:color=FF1493'' at 10pt

\newfontfamily\sans{DejaVu Sans}

%CV Sections inspired by:
%http://stefano.italians.nl/archives/26
\titleformat{\section}{\Large\scshape\raggedright\sffamily}{}{0em}{}[\titlerule]
%% \titlespacing{\section}{0pt}{-2pt}{0pt}

%-------------WATERMARK TEST [**not part of a CV**]---------------
\TPGrid[30mm,30mm]{30}{60}
%\setlength{\TPHorizModule}{30mm}
%\setlength{\TPVertModule}{\TPHorizModule}
%\textblockorigin{2mm}{0.65\paperheight}
\setlength{\parindent}{0pt}

\newcommand{\skill}{\textbf}
\newcommand{\skills}[2]{
  \item #2 #1
}

% \def\bullet{\textcolor{medg}{\symbol{"00BB}}}
\def\div{\,\textbar{}\,}

\usepackage{xspace}

% language macros

\newcommand{\lang}[2]{\expandafter\def\csname #1\endcsname{\skill{#2}\xspace}}

\lang{airflow}{Airflow}
\lang{android}{Android}
\lang{backbone}{Backbone.js}
\lang{bash}{Bash}
\lang{cascalog}{Cascalog}
\lang{ccpp}{C/C++}
\lang{clojurescript}{ClojureScript}
\lang{clojure}{Clojure}
\lang{cpp}{C++}
\lang{csh}{C\#}
\lang{c}{C}
\lang{django}{Django}
\lang{git}{git}
\lang{hadoop}{Hadoop}
\lang{haskell}{Haskell}
\lang{hive}{Hive}
\lang{html}{HTML/CSS}
\lang{java}{Java}
\lang{jquery}{jQuery}
\lang{js}{JavaScript}
\lang{linux}{Linux}
\lang{matlab}{MATLAB}
\lang{numpy}{NumPy}
\lang{opencv}{OpenCV}
\lang{php}{PHP}
\lang{postgres}{PostgreSQL}
\lang{python}{Python}
\lang{react}{React}
\lang{reagent}{Reagent}
\lang{reframe}{re-frame}
\lang{ruby}{Ruby}
\lang{salesforce}{Salesforce}
\lang{scala}{Scala}
\lang{scheme}{Scheme}
\lang{scipy}{SciPy}
\lang{spark}{Spark}
\lang{sql}{SQL}
\lang{teradata}{Teradata}
\lang{visb}{Visual Basic}


\begin{document}
\pagestyle{empty}

% \font\fb=''[cmr10]''

\par{\centering {\Huge Jarrell \textsc{Waggoner} }\bigskip\par}

%% \section{Biographical Data}
%%
%% \begin{tabular}{r p{3.5in}}
%%   \textsc{Address:}	& Department of Computer Science and Engineering, University of South Carolina, Columbia, SC 29208 \\
%%   \textsc{Phone:}     & 847-261-4747\\
%%   \textsc{email:}     & \href{mailto:malloc47@gmail.com}{malloc47@gmail.com} \\
%%   \textsc{Website:}	& \href{http://www.malloc47.com}{www.malloc47.com} \\
%%   \textsc{Citizenship:} & United States Citizen \\
%% \end{tabular}


\begin{multicols}{2}
\setlength{\parskip}{0pt}
  \section{Biographical}

\begin{tabularx}{\linewidth}{@{}l X@{}}
  \textsc{Address}     & \small{533 South 3rd Street, Suite 400} \\
                       & \small{c/o Rally, Minneapolis, MN 55415} \\
  \textsc{Phone}       & \href{tel:847-261-4747}{847-261-4747} \\
  \textsc{email}       & \href{mailto:jarrell.waggoner@gmail.com}{jarrell.waggoner@gmail.com} \\
\end{tabularx}

\vfill
\columnbreak

\section{Online}
\begin{tabularx}{\linewidth}{@{}l X@{}}
  \textsc{Website}     & \href{http://www.malloc47.com}{www.malloc47.com} \\
  \textsc{Twitter}     & \href{https://twitter.com/malloc47}{@malloc47} \\
  \textsc{github}      & \href{http://www.github.com/malloc47}{github.com/malloc47}\\
  \textsc{LinkedIn}    & \href{http://www.linkedin.com/in/malloc47}{linkedin.com/in/malloc47} \\
\end{tabularx}

\end{multicols}

\section{Research Interests}

computer vision, segmentation, contour completion, perceptual
grouping, document image analysis, event recognition, image
processing, artificial intelligence, pattern recognition \& machine
learning, data science, functional programming, GIS, Clojure

\section{Education}
\begin{tabularx}{\linewidth}{@{}r X l}
  \textsc{Aug. 2013} & Ph.D. in \textsc{Computer Science \& Engineering} & \textbf{University of South Carolina}\\
  & \small Advisor: Dr. Song \textsc{Wang} & \\%
  & \footnotesize Dissertation: ``Multi-Label Segmentation Propagation for Materials Science Images Incorporating Topology and Interactivity'' & \\[4ex]%
  \textsc{May} 2009 & Master of Engineering in \textsc{Computer Science} & \textbf{University of South Carolina}\\
  &\normalsize \textsc{GPA}: 3.8/4.0 | \small\emph{magna cum laude} \\[1ex]%
\end{tabularx}

\newcommand{\experience}[4]{
\textsc{#1} & #2 \\
\nopagebreak &\emph{#3}\\
\nopagebreak &\footnotesize{#4} \\
\nopagebreak \multicolumn{2}{c}{} \\ [-1ex]
}

\section{Industry Experience}
\vspace{-1em}
\newcommand{\industry}[4]{
\textsc{#1} & #2 &\emph{#3}\\
&\multicolumn{2}{p{14cm}}{\footnotesize{#4}}\\
\multicolumn{3}{c}{} \\ [-1ex]
}

\begin{longtable}{@{}p{2.2cm}|p{8cm} r}



\industry{2017\textemdash{}Present}%
  {Principal Software Engineer}%
  {\href{https://www.rallyhealth.com}{Rally Health, Inc.}}%
  {Individual contributor on the Data Infrastructure team, building
    tools and architecting data pipelines for Rally's analytics and
    reporting.
    \begin{itemize}[leftmargin=1.5em]
      \renewcommand\labelitemi{---}

    \item Architected complete rewrite of the data pipeline for the
      entire organization, moving from a fixed \skill{Cloudera}
      cluster with a \hive EDW to dynamic clusters with
      \skill{Databricks} with a \skill{Redshift} EDW fed by \spark
      ETLs written in \scala and scheduled with \skill{Airflow}.

    \item Coordinated with internal and external teams ranging from
      Ops to Data Science to define and build new data pipeline,
      spearheading the CI process, development and testing workflows,
      data ingestion process, EDW layout, monitoring/alerting, and
      data validation.

    \item Member of the Engineering Technical Staff, responsible for
      making cross-cutting engineering decisions, evaluating potential
      acquisitions, signing off on major architectural changes, and
      organizing technical interest groups.

    \item Technical lead of eight person engineering team; responsible
      for working with Product and Project managers to define and
      schedule work, standardize code review practices on the team,
      build consensus for new architectural approaches, reporting
      directly to VP-level management.\vspace*{-\baselineskip}
    \end{itemize}
  }

\industry{2016\textemdash{}2017}%
{Software Engineer}%
{\href{http://www.drw.com}{DRW Holdings, LLC}}%
{Member of the Trading Infrastructure team, developing the internal
  platform used across trading desks at DRW. Building greenfield
  high-performance service-oriented systems using \clojure and \java
  and maintaining legacy applications in \ruby and \csh among a
  catalog of over 50 microservices.

  \begin{itemize}[leftmargin=1.5em]
    \renewcommand\labelitemi{---}

  \item Contributed to an extensive reconciliation tool used to
    balancing cashflows for high-volume trading, written in \ruby.

  \item Extended a research workflow tool used for computing the value
    and settle price of options, futures, equities, and other
    financial instruments, written in \clojure.

  \item Developed and extended multiple UI frontends for internal
    tools using \skill{React} and
    \skill{Reagent}.\vspace*{-\baselineskip}
  \end{itemize}
}

\industry{2013\textemdash{}2016}%
{Senior Software Engineer}%
{\href{http://www.groupon.com}{Groupon, Inc.}}%%
{Engineer on the Flux team building Data Science pipelines,
  contributed to Project Genesis which filled \salesforce with 250K
  new leads from crawled data sources, and Tech Lead of the Supply
  Intelligence team creating internal sales tools to optimize
  Groupon's supply funnel. Wrote production \clojure to develop
  service-oriented and big data systems.

  \begin{itemize}[leftmargin=1.5em]
    \renewcommand\labelitemi{---}

  \item Built a \postgres-backed high-performance caching and write
    management system around the \salesforce API that hits 10K
    req/min.

  \item Managed critical business automation of the sales lead
    assignment process that previously required an estimated 80 sales
    managers to conduct manually; led the effort to rearchitect this
    legacy system from an ad-hoc job scheduling platform written in
    \ruby and \bash to a multi-staged \hadoop pipeline written in
    \clojure to handle over 6M accounts.

    %% \item Operationalized \python within the organization by spinning
    %%   up an internal PyPI server; introducing a pex-driven
    %%   single-artifact deployment process; creating an interest group,
    %%   mailing list, and internal wiki; and standardized configuration,
    %%   logging, and resource management.

    %% \item Developed an end-to-end system and coordinated with product
    %%   and business teams to operationalize 250K leads in \salesforce
    %%   from scraped web data.

  \item Built out an ETL management and machine learning platform
    using \python, \clojure, \hive, and \spark to run mission-critical
    Decision Tree Learning models to predict customer attrition,
    lifetime customer value, and merchant value.

  \item Oversaw technical decisions, engaged in mentorship,
    established best practices, coordinated with stake holders, and
    led multiple major technical initiatives on a team of 5
    developers. \vspace*{-\baselineskip}
  \end{itemize}
}

\industry{2012\textemdash{}2014}%
{Technical Lead}%
{\href{http://www.terrastride.com/}{TerraStride, Inc.}}%
{Software developer in an agile startup environment creating the
  \href{http://www.huntstand.com}{huntstand.com} web
  application. Written using \python, \django, and \backbone; deployed
  to \skill{AWS}.  Responsible for curating full technology stack and
  coordinating with $5$ developers.}

\industry{2011\textemdash{}2013}%
{Project Manager}%
{Palmetto Computer Labs}%
{Assisted in planning the POSSCON conference. Managed the Open IT Lab
  and associated projects (Android Development). Provided software
  support for websites and managed projects.}

\industry{2011}%
{Contractor}%
{Elastic Vision Consulting}%
{Created a parser and generator for XML medical records formats (CCR
  and CCD) in Java using JDOM, JAXB, SAX, Xerces, and Hibernate
  (HSQLDB), on an Axis2+Jetty6 driven server.}

%% \industry{2005}%
%% {Intern --- Technical Writer}%
%% {JAARS, Inc.}%
%% {Created documentation and integrated context-sensitive online help
%%   system for speech and linguistic software written in C++ and Visual
%%   Basic.}

%% \industry{2001\textemdash{}2002}%
%% {Volunteer Software Developer}%
%% {JAARS, Inc.}%
%% {Spearheaded the conversion from VB4 to VB6 for the linguistic
%%   reference tool
%%   \href{http://www.sil.org/computing/ipahelp/ipaprvw2.htm}{IPA Help}.}

\end{longtable}

\section{Research Experience}

\begin{longtable}{@{}p{2.4cm}|p{13.6cm}}

\experience{2011---2013}%
{Research Assistant funded by \textsc{AFOSR}}%
{Materials Volume Segmentation}%
{Developed segmentation methods for materials image
  volumes in \emph{Python+NumPy/SciPy} and \emph{MATLAB} at the
  \textsc{Computer Vision Lab} at \textsc{USC}. Managed the lab
  computer network and organized weekly lab meetings.  Created GUI
  interface using wxWidgets for assisted segmentation, and conducted
  large-scale evaluations on multiple datasets for metallic and
  biological materials.}

\experience{2010---2011}%
{Research Assistant funded by \textsc{DARPA}}%
{Video Event Recognition}%
{Explored segmentation methods for video event
  recognition. Attended P.I. meetings in San Diego (2010) and
  Colorado (2011). Developed algorithms in \emph{Scheme} to process
  a corpus of thousands of videos extracted into over 3 million
  frames using a high-performance computing cluster.}

\experience{2009---2010}%
{NEH Fellow at the \textsc{Center for Digital Humanities}}%
{Digital Collation}%
{Created a \textsc{digital collation} application to
  handle automatic differencing of sub-textual inconsistencies among
  multiple copies of \emph{The Faerie Queene} by \textsc{Edmund
    Spenser} in \emph{MATLAB} to process tens of thousands of book
  page images.}

\end{longtable}

\section{Teaching Experience}
\vspace{-1em}

\begin{longtable}{@{}p{2.4cm}|p{13.6cm}}

\experience{2008--2009}%
{GK-12 Fellow at \textsc{Crayton Middle School}}%
{8\textsuperscript{th} Grade Science}%
{Served in Crayton Middle School, coordinating with the classroom instructor to enhance the science curriculum and activities in an 8\textsuperscript{th} grade science classroom. Subsequently coordinated and taught at the \textsc{GK-12 Institute for Teachers}, presenting the activities developed and delivered in the classroom.}

\experience{2007--2008, 2011}%
{Graduate Teaching Assistant at \textsc{USC}}%
{Web Development}%
{Supervised CSCE~145 labs, covering software development with \textsc{Java}, and taught CSCE~102, covering \textsc{Javascript}, \textsc{HTML}, and \textsc{CSS}. Taught~CSCE~211 covering digital logic design.}

\experience{Spring 2007}%
{Instructor for \textsc{CSCE 204} at \textsc{USCL}}%
{Introductory Programming}%
{Hired as special faculty. Taught introductory Visual Basic for majors and non-majors. Selected textbooks, developed all course material, graded all assignments. Worked with Dr. Noni M. Bohonak}

\experience{Fall 2006}%
{Camp Instructor for \textsc{USCL Arts and Sciences Adventure Camp}}%
{5\textsuperscript{th}-8\textsuperscript{th} Grade Students}%
{Worked in collaboration with Dr. Dwayne Brown. One of two instructors teaching Math and Computer Science to grade school students.}

\experience{2003--2007}%
{Professional Tutor at \textsc{USCL Academic Success Center}}%
{High School and College Students}%
{Student and graduate tutor for college-level Mathematics, Computer Science, Physics, and English classes.}

\end{longtable}

\pagestyle{myheadings}
\markright{Jarrell Waggoner}


\let\originalbibitem\bibitem
\def\bibitem#1#2\par{%
  \noexpandarg
  \originalbibitem{#1}
  \StrSubstitute{#2}{Jarrell Waggoner}{\textbf{Jarrell Waggoner}}\par}

%\section{Publications}
\nocite{derrick:16}
\nocite{waggoner:15}
\nocite{zhou:14}
\nocite{waggoner:phd}
\nocite{waggoner:14}
\nocite{waggoner:13a}
%% \nocite{waggoner:13b}
\nocite{waggoner:13c}
\nocite{waggoner:11}
\nocite{wang:11}
\nocite{temlyakov:10}
\nocite{zhang:10}
\nocite{waggoner:12}
\nocite{barbu:12}
\nocite{barbu:12b}
\nocite{zhang:12}
\nocite{temlyakov:13}
\nocite{salvi:13a}
\nocite{salvi:13b}

\renewcommand\refname{Publications}
\bibliography{cv}
\bibliographystyle{plainyr-rev}

\section{Posters/Presentations}
\begin{enumerate}
\renewcommand{\labelenumi}{[P\arabic{enumi}] }
\item Rules Engines: Logic As Data Structure. \emph{Palmetto Open Source Software Conference}. Columbia, SC. April 14, 2015.
\item Python for Computer Vision. \emph{All Things Open}. Raleigh, SC. October 24, 2013.
\item Interactive Grain Image Segmentation Using Graph Cut Algorithms. \emph{USC Graduate Student Day}. Columbia, SC. April 12, 2013.
\item Extending Django.  \emph{Palmetto Open Source Software
    Conference}.  Columbia, SC.  March 28, 2013.
\item Computer Science: Research, Industry, and Entrepreneurship.
  \emph{Careers in Science Lecture Series}.  Lancaster, SC.  March 6,
  2013.
\item Interactive Grain Image Segmentation Using Graph Cut Algorithms.
  \emph{SPIE (Computational Imaging XI)}.  Burlingame, CA.  February
  6, 2013.
\item Homeomorphic Multi-Structure Propagation for Metallic Image
  Segmentation.  \emph{Gamecock Computing Research Symposium}.
  Columbia, SC.  October 5, 2012.
\item Android Application Development Workshop.  \emph{Appathon
    Contest}.  Columbia, SC.  November 17, 2012.
\item Open Source and Education. \emph{SC Municipal Technology
    Association (SCMTA) Conference}. Charleston, SC.  September 6,
  2012.
\item Open Source and Higher Education.  \emph{SC Technical College
    System (SCTCS) Conference}.  Columbia, SC.  September 25, 2012.
\item Introduction to Android Development.  \emph{Digital Humanities
    High Performance Computing (DHHPC) Workshop}.  Columbia, SC.
  August 8, 2012.
\item Combining Global Labeling and Local Relabeling for Metallic
  Image Segmentation.  \emph{SPIE (Computational Imaging X)}.
  Burlingame, CA.  January 23, 2012.
\item Open Source and Government.  \emph{SC Government Management
    Information Systems (SCGMIS) Software Developers Workshop}.
  Columbia, SC.  January 19, 2012.
\item Superpixel Contour Completion.  \emph{DARPA Mind's Eye PI
    Meeting}.  Denver, CO.  January 20, 2011.
\end{enumerate}

%% \section{Guest Lectures}
%% \begin{enumerate}
%% \renewcommand{\labelenumi}{[G\arabic{enumi}] }
%% %% \item \emph{Combining Global Labeling and Local Relabeling for Metallic Image Segmentation}. Graduate Student Day Competition, Second Place. April 8, 2011.
%% %% \item \emph{Image Registration for Digital Collation}. Graduate Student Day Competition, Honorable Mention. April 2, 2010.
%% \item \emph{Building Chrome Extensions}.  In CSCE 242.  Guest lecture for Dr. José M. Vidal.  November~30, 2012.
%% \item \emph{Modeling in Blender}.  In CSCE 552.  Guest lecture for Dr. Jijun Tang.  February~28, 2011.
%% \item \emph{Aspect-Oriented Programming}. In CSCE 531. Guest lecture for Dr. Marco Valtorta. March 19, 2008.
%% \item \emph{Math 241}. Vector Calculus. Guest lecture for Dr. Dwayne Brown. April~23---26, 2007.
%% \item \emph{Math 242}. Differential Equations. Guest lecture for Dr. Dwayne Brown. April~23---26, 2007.
%% \end{enumerate}

%\section{Travel}
%\begin{enumerate}
%\renewcommand{\labelenumi}{[T\arabic{enumi}] }
%\item \emph{Mind's Eye PI Meeting}. DARPA Project P.I.~meeting. Denver, CO. January 20---21, 2011.
%\item \emph{Tomography and its Applications to Materials Science and Non-Destructive Evaluation}. Organizd by M. De Graef, L. Drummy, J. Simmons, M. Comer, C. Bouman, and J. Knopp. Tech\^{}Edge, Dayton, Ohio. December 13---15, 2010.
%\item Visiting scholar. In collaboration with J. M. Siskind. Purdue University, West Lafayette, IN. December 6---22, 2010 \& January 5---16, 2011.
%\item \emph{Mind's Eye Kickoff Meeting}. DARPA Project P.I.~meeting. San Diego, CA. September 23---24, 2010.
%\end{enumerate}

\section{Honors/Awards}
\begin{center}
\begin{tabular*}{0.75\textwidth}{rll}
2012 & Gamecock Computing Research Symposium Poster Session,  First Place & \multirow{5}{*}{{\lighttext \textcolor{lightg}{\begin{turn}{-90}USC\end{turn}}}}\\
     & Graduate Student Day Presentation,  First Place \\
2011 & Graduate Student Day Presentation,  Second Place \\
2010 & Graduate Student Day Presentation,  Honorable Mention \\
2009 & Upsilon Pi Epsilon \\
% \multicolumn{3}{r}{}\\
% 2006 & Senior Computer Science Award & {\lighttext \textcolor{lightg}{Bryan College}}\\
\multicolumn{3}{r}{}\\
2004 & Clara P. Hammond Award & \multirow{3}{*}{{\lighttext \textcolor{lightg}{\begin{turn}{-90}USCL\end{turn}}}} \\
& Science and Mathematics Award \\
& Highest Academic Average Award \\
\end{tabular*}
\end{center}

%\section{Professional Societies}

\section{Classes Taught}
\vspace{-1em}
\begin{center}
%\begin{tabular*}{0.75\textwidth}{r @{\hspace{0.5em}\textcolor{lightg}{\symbol{"00BB}}\hspace{0.5em}} l @{\extracolsep{\fill}} l }
\begin{tabular*}{0.75\textwidth}{r @{\hspace{0.5em}\textcolor{lightg}{\symbol{"00BB}}\hspace{0.5em}} l l c }
%\multicolumn{3}{r}{University of South Carolina}\\
2012--2013 & Open Source 101 & Open Source Software & \multirow{3}{*}{{\lighttext \textcolor{lightg}{\begin{turn}{-90}IT-oLogy\end{turn}}}}  \\
2012--2013 & Version Control 101 & git, github \\
2012--2013 & Command Line 101 & Linux, BASH \\
\multicolumn{3}{r}{}\\
Fall 2011 & CSCE 211 & Digital Logic Design & \multirow{4}{*}{{\lighttext \textcolor{lightg}{\begin{turn}{-90}USC\end{turn}}}} \\
Summer II 2008 & CSCE 102 & HTML/CSS/JavaScript \\
Spring 2008 & CSCE 145 Lab & Java \\
Fall 2007 & CSCE 145 Lab & Java \\
\multicolumn{3}{r}{}\\
%\multicolumn{3}{r}{University of South Carolina at Lancaster}\\ \hline
Spring 2007 & CSCE 204 & Visual Basic & \multirow{2}{*}{{\lighttext \textcolor{lightg}{\begin{turn}{-90}USCL\end{turn}}}} \\
Spring 2007 & Math 241 \& Math 242 & Maple \\
\end{tabular*}
\end{center}

\section{Service}
\vspace{-1em}
\newcommand{\service}[2]{
  \textsc{#1} & #2\\
  %% \multicolumn{2}{c}{} \\
}

\renewcommand*{\arraystretch}{1.33}
\begin{longtable}{r|p{10cm}}
  \service{Book Reviewer}{\href{https://www.packtpub.com/big-data-and-business-intelligence/practical-data-analysis}{Practical Data Analysis}, Packt Publishing, 2013}
  \service{Mentoring}{Groupon internship program, 2014}
  \service{Webmaster}{\href{http://cvl.cse.sc.edu/wvm2013/}{Winter Vision Meetings, 2013}}
  \service{Webmaster}{\href{http://cvl.cse.sc.edu/wacv2013/}{Workshop on the Applications of Computer Vision, 2013}}
  \service{Judge}{Discovery Day --- Undergraduate Research Presentations}
  \service{Reviewer}{Pattern Recognition Letters}
  \service{Reviewer}{IEEE Transactions on Pattern Analysis and Machine Intelligence}
  %% \service{Fellow}{NSF GK-12 Program}
  %% \service{Member}{Institute of Electrical and Electronics Engineers (IEEE)}
  \service{SysAdmin}{Computer Vision Lab}
\end{longtable}

\vspace{2em}

\section{Personal and Open Source Projects}
\newcommand{\proj}[3]{
  \textsc{#1} & #2\\
   &\href{http://www.#3}{#3}\\
   \multicolumn{2}{c}{} \\ [-1.5ex]
}

\newcommand{\projlh}[4]{
  \textsc{#1} & #2\\
   &\href{#3}{#4}\\
}

\begin{longtable}{@{}p{3cm}|p{13cm}}

  \proj{matsciseg}%
  {Framework for propagated 3D volume segmentation, used in my dissertation work.  Algorithms created in \python and \cpp and exposed as a web API using \django. Includes a web application that consumes the API created in \js, and \jquery.}%
  {github.com/malloc47/matsciseg}

%%   \proj{\href{http://nonpartisan.me}{nonpartisan.me}}%
%% {Google Chrome extension that filters social media websites for political keywords.  Available in the \href{https://chrome.google.com/webstore/detail/nonpartisanme/ninebcppidndhampaggnjbijpacoadgg}{Chrome Web Store}.  Featured in the \href{http://www.charlestoncitypaper.com/charleston/sick-of-politics-on-facebook-try-this-browser-tool/Content?oid=4153447}{Charleston City Paper}.}%
%% {github.com/malloc47/nonpartisan.me}

%% \proj{term-do}{An interactive terminal prompt that displays potential command completions as you type.  A hybrid of gnome-do and Emacs's ido-mode.  Works on many tested VT100 terminal types; built in~\skill{C++}.  Includes client/server architecture implemented with boost.interprocess and full-featured plugin system.
%%   %Available in the \href{https://aur.archlinux.org/packages/term-do-git/}{Arch Linux AUR}.
%% }{github.com/malloc47/term-do}

  \proj{Ratio Contour}{Maintainer and contributor for the Ratio Contour project, a salient object detection and segmentation method used for computer vision applications.  Developed in \skill{C} and \skill{MATLAB}.}{github.com/malloc47/ratio-contour}

  \proj{Digital Collation}{Research project to ``collate'' high-resolution documents by using image registration, accomplished using the SIFT feature detector and a thin plate spline warping technique, written in MATLAB.}{github.com/malloc47/digital-collation}

  % \proj{PMLDAP}{\skill{Linux} user management tool for Linux clusters.  Created as a simplified replacement for LDAP.  Capable of bootstrapping new systems, synchronizing users and configuration files, and running distributed commands.  Written in \skill{Bash}.}{github.com/malloc47/pmldap}

  \proj{matscicut}{An energy minimization framework for segmenting 3D materials volumes. Prototype of dissertation work, created in C++ using OpenCV libraries, with assorted MATLAB helper utilities.}{github.com/malloc47/matscicut}

  % \proj{git-hq}{A remote management system for git, created in Python.}{github.com/malloc47/git-hq}

  %% \proj{Sina Weibo Mobile Client}{Created a \skill{J2ME}-based prototype mobile client for the popular Chinese \institution{Sina} microblogging service, similar to \institution{Twitter}.  Targeted at limited-functionality CLDC phones and uses a custom \skill{Java} wrapper for the \institution{Sina} API.  Employs symmetric-key encryption for personal data.}{bd.weibo.10086.cn/2012/downloads\_kjav}

  \proj{befunge.py}{Complete \href{https://en.wikipedia.org/wiki/Befunge}{Befunge} interpreter written in \python.  Implements the Befunge 93 specification, and is one of the closest Python equivalents to the \c reference implementation.}{github.com/malloc47/befunge.py}

\end{longtable}

\begin{minipage}{\linewidth}
\newcommand{\skills}[2]{
  \item #2 #1
}
\section{Skills \& Languages}
\begin{multicols}{4}
\raggedcolumns
\begin{itemize}
\renewcommand{\labelitemi}{}
\renewcommand{\skill}{\textnormal}
\setlength{\itemsep}{1pt}
\setlength{\parskip}{0pt}
\setlength{\parsep}{0pt}


\skills{\bash}{\threeskill}
\skills{\ccpp}{\twoskill}
\skills{\clojure}{\threeskill}
\skills{\django}{\twoskill}
\skills{Emacs Lisp}{\twoskill}
\skills{\git}{\threeskill}
\skills{GNU/\linux}{\threeskill}
\skills{\hadoop}{\threeskill}
\skills{\haskell}{\oneskill}
\skills{\java}{\threeskill}
\skills{\js}{\twoskill}
\skills{\LaTeX}{\twoskill}
\skills{\matlab}{\twoskill}
\skills{\numpy/\scipy}{\threeskill}
\skills{\opencv}{\threeskill}
\skills{\postgres}{\threeskill}
\skills{\python}{\threeskill}
\skills{\scala}{\threeskill}
\skills{\scheme}{\threeskill}
\skills{\spark}{\threeskill}


% graveyard

%% \skills{Blender}{\twoskill}
%% \skills{English}{\threeskill}
%% \skills{LAMP Stack}{\fournotes}
%% \skills{LISP}{\onenote}
%% \skills{Learning}{\fournotes Machine}
%% \skills{MS Office}{\fivenotes}
%% \skills{Maple}{\twoskill}
%% \skills{Networking}{\threenotes}
%% \skills{Processing}{\fivenotes Image}
%% \skills{Sys. Admin.}{\threenotes}
%% \skills{Visual Basic}{\fivenotes}
%% \skills{Windows}{\fivenotes}
%% \skills{Wordpress}{\fournotes}
%% \skills{\hive}{\oneskill}
%% \skills{\html}{\threeskill}
%% \skills{\jquery}{\twoskill}
%% \skills{\php}{\oneskill}

\end{itemize}
\end{multicols}
\begin{footnotesize}
  \oneskill Small-scale or personal projects \hfill
  \twoskill Used in production  \hfill
  \threeskill Used in large-scale production systems
\end{footnotesize}
\end{minipage}

\section{Activities}
teaching, open source software, GIS visualization, Linux,
\href{https://soundcloud.com/malloc47}{music composition}

\null\vfill
\footnotesize{
  Online:  \href{http://cv.malloc47.com}{cv.malloc47.com} \hfill
  Résumé: \href{http://resume.malloc47.com}{resume.malloc47.com} \hfill
  Source:  \href{https://github.com/malloc47/cv/tree/master}{github.com/malloc47/cv/}
}

%%\XeTeXpdffile ''cv.pdf'' page 1 scaled 800

\end{document}
