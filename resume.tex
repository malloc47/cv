\documentclass[10pt]{article}

\usepackage{xunicode,xltxtra,url,parskip}
\usepackage[usenames,dvipsnames]{xcolor}
\usepackage[left=1in, right=1in, top=0.5in, bottom=0.5in]{geometry}
\usepackage{titlesec}
\usepackage{multicol}
\usepackage{xstring}
\usepackage{enumitem}
\usepackage{changepage}
\usepackage{mdframed}
\usepackage[xetex,
            unicode,
            pdfencoding=auto,
            pdfinfo={
              Title={malloc47/resume},
              Author={Jarrell Waggoner},
              Subject={Jarrell Waggoner Résumé},
              Keywords={computer vision, image processing, artificial intelligence, pattern recognition, machine learning, data science, data engineering, functional programming, web development, GIS, clojure},
              Producer={xelatex},
              Creator{xelatex}
            },
            ]{hyperref}
\usepackage{textpos}
\usepackage{tabularx}

\setlength{\columnsep}{0pt}

% \makeatletter
% \renewcommand*{\@biblabel}[1]{\hfill[#1]}
% \makeatother

% Template obtained from http://www.cv-templates.info/2009/03/professional-cv-latex/

%Setup hyperref package, and colours for links
\definecolor{linkcolour}{rgb}{0,0.2,0.6}
\hypersetup{colorlinks,breaklinks,urlcolor=linkcolour, linkcolor=linkcolour}

%Color
\definecolor{lightg}{HTML}{999999}
\definecolor{medg}{HTML}{666666}
\definecolor{darkg}{HTML}{333333}

% Bullets
\definecolor{noteone}{HTML}{999999}
\definecolor{notetwo}{HTML}{848484}
\definecolor{notethree}{HTML}{424242}
\definecolor{notefour}{HTML}{212121}
\definecolor{notefive}{HTML}{000000}

\newcommand{\fivenotes}{%
	\textcolor{noteone}{\symbol{"2022}}
	\textcolor{notetwo}{\symbol{"2022}}
	\textcolor{notethree}{\symbol{"2022}}
	\textcolor{notefour}{\symbol{"2022}}
	\textcolor{notefive}{\symbol{"2022}}
}
\newcommand{\fournotes}{%
	\textcolor{noteone}{\symbol{"2022}}
	\textcolor{notetwo}{\symbol{"2022}}
	\textcolor{notethree}{\symbol{"2022}}
	\textcolor{notefour}{\symbol{"2022}}
	\textcolor{white}{\symbol{"2022}}
}
\newcommand{\threenotes}{%
	\textcolor{noteone}{\symbol{"2022}}
	\textcolor{notetwo}{\symbol{"2022}}
	\textcolor{notethree}{\symbol{"2022}}
	\textcolor{white}{\symbol{"2022}}
	\textcolor{white}{\symbol{"2022}}
}
\newcommand{\twonotes}{%
	\textcolor{noteone}{\symbol{"2022}}
	\textcolor{notetwo}{\symbol{"2022}}
	\textcolor{white}{\symbol{"2022}}
	\textcolor{white}{\symbol{"2022}}
	\textcolor{white}{\symbol{"2022}}
}
\newcommand{\onenote}{%
	\textcolor{noteone}{\symbol{"2022}}
	\textcolor{white}{\symbol{"2022}}
	\textcolor{white}{\symbol{"2022}}
	\textcolor{white}{\symbol{"2022}}
	\textcolor{white}{\symbol{"2022}}
}

%FONTS
% \defaultfontfeatures{Mapping=tex-text}
% \setmainfont[SmallCapsFont = Fontin SmallCaps]{Fontin}

\setromanfont [Ligatures={Common}, BoldFont={Linux Libertine Bold}, ItalicFont={Linux Libertine Italic}]{Linux Libertine}
\setsansfont [Ligatures={Common}, BoldFont={GeosansLight}, ItalicFont={GeosansLight}]{GeosansLight}
\setmonofont{GeosansLight} 

\font\lighttext=''Baskerville-Normal:color=787878'' at 10pt
\font\lighttextweb=''Baskerville-Normal:color=FF1493'' at 10pt

%CV Sections inspired by: 
%http://stefano.italians.nl/archives/26
\titleformat{\section}{\Large\scshape\raggedright}{}{0em}{}[\titlerule]
\titlespacing{\section}{0pt}{3pt}{3pt}
%Tweak a bit the top margin

%\addtolength{\voffset}{-1.3cm}

%-------------WATERMARK TEST [**not part of a CV**]---------------
\TPGrid[30mm,30mm]{30}{60}
%\setlength{\TPHorizModule}{30mm}
%\setlength{\TPVertModule}{\TPHorizModule}
%\textblockorigin{2mm}{0.65\paperheight}
\setlength{\parindent}{0pt}

\titlespacing{\section}{0pt}{-2pt}{0pt}

\begin{document}
\pagestyle{empty}
% \font\fb=''[cmr10]''

\par{\centering {\Huge Jarrell \textsc{Waggoner} }\bigskip\par}

\begin{multicols}{2}
\setlength{\parskip}{0pt}
\section{Biographical}

\begin{tabularx}{\linewidth}{@{}l X@{}}
  \textsc{Address} & \small{533 South 3rd Street, Suite 400} \\
                   & \small{c/o Rally, Minneapolis, MN 55415} \\
  \textsc{Phone}   & \href{tel:847-261-4747}{847-261-4747}\\
  \textsc{email}   & \href{mailto:jarrell.waggoner@gmail.com}{jarrell.waggoner@gmail.com} \\
\end{tabularx}

\vfill
\columnbreak

\section{Online}
\begin{tabularx}{\linewidth}{@{}l X@{}}
  \textsc{Website}	& \href{http://www.malloc47.com}{www.malloc47.com} \\
  \textsc{Twitter}     & \href{https://twitter.com/malloc47}{@malloc47} \\
  \textsc{github}      & \href{http://www.github.com/malloc47}{github.com/malloc47}\\
  \textsc{LinkedIn}    & \href{http://www.linkedin.com/in/malloc47}{linkedin.com/in/malloc47} \\
\end{tabularx}

\end{multicols}

\begin{tabularx}{\textwidth}{@{}l X}
  \textsc{Interests} & computer vision, image processing, artificial
  intelligence, pattern recognition \& machine learning, data science,
  data engineering, functional programming, web development, GIS,
  Clojure
\end{tabularx}

\newcommand{\degree}[4]{\textsc{#1} & \textbf{#2} & \textsc{#3} & \textbf{#4}\\}

%% \vspace{0em}

\section{Education}
\begin{tabular*}{\textwidth}{@{\extracolsep{\fill}}l l p{5.5cm} r}

  \degree{Aug. 2013}%
  {Ph.D.}%
  {Computer Science \& Engineering}%
  {University of South Carolina}

  \degree{May 2009}%
  {M.E.}%
  {Computer Science \& Engineering}%
  {University of South Carolina}

\end{tabular*}

\newmdenv[
  topline=false,
  bottomline=false,
  rightline=false,
  leftmargin=-1ex,
  rightmargin=0,
  skipabove=0,
  skipbelow=0,
  innerrightmargin=0,
  innerleftmargin=1ex,
  innertopmargin=0,
  innerbottommargin=0
]{sideblock}

\newcommand{\experience}[5]{
  \begin{adjustwidth}{1in}{}
    \begin{sideblock}
      \begin{textblock*}{1in}(-1in,0.65em)
        \textsc{#1}
      \end{textblock*}
      \textbf{#2} #3 \textsc{#4}
      \vspace{-1ex}

      \small{#5}
    \end{sideblock}
  \end{adjustwidth}
  \vspace{\baselineskip}
}

\newenvironment{exlist}
  {\begin{itemize}[
      leftmargin=1.5em,
      topsep=-0.5em,
      itemsep=0pt,
      parsep=0.5ex,
      partopsep=0pt,
    ]
      \renewcommand\labelitemi{---}}
  {\end{itemize}}

\section{Experience}

\vspace{0.75em}

\experience{2017---Present}%
{Principal Software Engineer}%
{at}%
{\href{https://www.rallyhealth.com}{Rally Health, Inc.}}%
{Individual contributor on the Data Infrastructure team, building
  tools and architecting data pipelines for Rally's analytics and
  reporting.

  \begin{exlist}
  \item Architected complete rewrite of the data pipeline for the
    entire organization, moving from a fixed \skill{Cloudera}
    cluster with a \hive EDW to dynamic clusters with
    \skill{Databricks} with a \skill{Redshift} EDW fed by \spark
    ETLs written in \scala and scheduled with \skill{Airflow}.

  \item Coordinated with internal and external teams ranging from
    Ops to Data Science to define and build new data pipeline,
    spearheading the CI process, development and testing workflows,
    data ingestion process, EDW layout, monitoring/alerting, and
    data validation.

  \item Member of the Engineering Technical Staff, responsible for
    making cross-cutting engineering decisions, evaluating potential
    acquisitions, signing off on major architectural changes, and
    organizing technical interest groups.

  \item Technical lead of eight person engineering team; responsible
    for working with Product and Project managers to define and
    schedule work, standardize code review practices on the team,
    build consensus for new architectural approaches, reporting
    directly to VP-level management.%% \vspace*{-\baselineskip}
  \end{exlist}
}

\experience{2016---2017}%
{Software Engineer}%
{at}%
{\href{http://www.drw.com}{DRW Holdings, LLC}}%
{Member of the Trading Infrastructure team, developing the internal
  platform used across trading desks at DRW. Building greenfield
  high-performance service-oriented systems using \clojure and \java
  and maintaining legacy applications in \ruby and \csh among a
  catalog of over 50 microservices.

  \begin{exlist}

    \item Contributed to an extensive reconciliation tool used to
      balancing cashflows for high-volume trading, written in \ruby.

    \item Extended a research workflow tool used for computing the
      value and settle price of options, futures, equities, and
      other financial instruments, written in \clojure.

    \item Developed and extended multiple UI frontends for internal
      tools using \skill{React} and
      \skill{Reagent}.
  \end{exlist}
}

\experience{2013---2016}%
{Senior Software Engineer}%
{at}%
{\href{http://www.groupon.com}{Groupon, Inc.}}%
{Engineer on the Flux team building Data Science pipelines,
  contributed to Project Genesis which filled \salesforce with 250K
  new leads from crawled data sources, and Tech Lead of the Supply
  Intelligence team creating internal sales tools to optimize
  Groupon's supply funnel. Wrote production \clojure to develop
  service-oriented and big data
  \begin{exlist}

  \item Built a \postgres-backed high-performance caching and write
    management system around the \salesforce API that hits 10K
    req/min.

  \item Managed critical business automation of the sales lead
    assignment process that previously required an estimated 80
    sales managers to conduct manually; led the effort to
    rearchitect this legacy system from an ad-hoc job scheduling
    platform written in \ruby and \bash to a multi-staged \hadoop
    pipeline written in \clojure to handle over 6M accounts.

  %% \item Operationalized \python within the organization by spinning
  %%   up an internal PyPI server; introducing a pex-driven
  %%   single-artifact deployment process; creating an interest group,
  %%   mailing list, and internal wiki; and standardized configuration,
  %%   logging, and resource management.

  %% \item Developed an end-to-end system and coordinated with product
  %%   and business teams to operationalize 250K leads in \salesforce
  %%   from scraped web data.

  \item Built out an ETL management and machine learning platform
    using \python, \clojure, \hive, and \spark to run
    mission-critical Decision Tree Learning models to predict
    customer attrition, lifetime customer value, and merchant
    value.

  \item Oversaw technical decisions, engaged in mentorship,
    established best practices, coordinated with stake holders, and
    led multiple major technical initiatives on a team of 5
    developers.
  \end{exlist}
}

\experience{2012---2014}%
{Technical Engineer}%
{at}%
{\href{http://www.terrastride.com/}{Terrastride, Inc.}}%
{Software developer in an agile startup environment creating the
  \href{http://www.huntstand.com}{huntstand.com} web application.
  Written using \python, \django, and \backbone; deployed to
  \skill{AWS}.  Responsible for curating full technology stack and
  coordinating with $5$ developers.}

\experience{2011---2013}%
{Research Assistant}%
{at}%
{USC \href{http://cvl.cse.sc.edu/}{Computer Vision Lab}}%
{Dissertation research on computer vision models and algorithms for
  materials science image segmentation in \python, \numpy, \scipy,
  \opencv, and \matlab.  Created a web interface using \django, \js,
  and \jquery. Conducted large-scale analysis using a 98-core
  high-performance computing system.}

%% \experience{2011---2013}%
%% {Project Manager}%
%% {at}%
%% {\href{http://palmettocomputerlabs.com/}{Palmetto Computer Labs}}%
%% {Created and taught workshops on \git, the \linux command line,
%%   \android development, and open source software for hundreds
%%   of students, developers, and government officials at
%%   \institution{\href{http://it-ology.org/}{IT-oLogy}}.  Managed the
%%   \institution{\href{http://open-it-lab.com/}{Open IT Lab}} and
%%   associated projects. Assisted in planning
%%   \institution{\href{http://posscon.org/}{POSSCON}}.}

%% \experience{2011}%
%% {Contractor}%
%% {for}%
%% {Elastic Vision Consulting}%
%% {Built a parser and generator for \skill{XML} medical records
%%   formats (CCR and CCD) in a \java web application.  Written
%%   using \skill{JDOM}, \skill{Xerces}, and \skill{Hibernate}, and run
%%   on an \skill{Axis2+Jetty6} driven server.}

\experience{2010---2011}%
{Research Assistant}%
{for the}%
{DARPA
  \href{https://en.wikipedia.org/wiki/Mind\%27s_Eye_(US_military)}{Mind's
    Eye Program} }%
{Researched video event recognition for the DARPA Mind's Eye
  program.  Collaborated with $10$ students and faculty members
  across three institutions.  Developed algorithms in
  \scheme, \bash, \matlab, and \c to
  process a corpus of 3480 videos extracted into over 1.5 million
  frames. Distributed processing over $7$ HPC machines.
  \href{http://0xab.com/research/video-in-sentences-out.html}{0xab.com/research/video-in-sentences-out.html}
  ,
  \href{https://www.github.com/malloc47/video-in-sentences-out}{github.com/malloc47/video-in-sentences-out}}

\experience{2009---2010}%
{NEH Fellow}%
{at the}%
{\href{https://sc.edu/about/centers/digital_humanities/index.php}{USC Center for Digital Humanities}
  (Sapheos/\href{http://sc.edu/about/centers/digital_humanities/projects/paragon.php}{Paragon}
  Project)}%
{Developed the prototype for a \emph{digital collation} application
  to identify sub-textual inconsistencies among multiple copies of
  \emph{The Faerie Queene} by \textsc{Edmund Spenser}.  Created in
  \matlab using \skill{VLFeat} and \opencv to process tens of
  thousands of book page images.
  \href{https://www.github.com/malloc47/digital-collation}{github.com/malloc47/digital-collation}}

%% \experience{2007---2011}%
%% {Teaching Assistant}%
%% {for}%
%% {\href{https://www.cse.sc.edu/}{USC Department of Computer Science
%%     and Engineering}}%
%% {Taught classes in software development, web development, and
%%   computer engineering, covering \java, \js, \html, and
%%   \skill{Visual Basic}.  Created syllabi and course objectives,
%%   developed and graded projects and assignments, supervised labs,
%%   and tutored students.}

%% \experience{2005}%
%% {Intern---Technical Writer}%
%% {at}%
%% {JAARS, Inc.}%
%% {Created documentation and integrated context-sensitive online help
%%   system for speech and linguistic software written in C++ and
%%   Visual Basic.}

% \experience{2001---2002}%
% {Volunteer Software Developer}%
% {at}%
% {JAARS, Inc.}%
% {Spearheaded the conversion from \skill{Visual Basic 4} to
%   \skill{Visual Basic 6} for the linguistic reference tool
%   \href{http://www.sil.org/computing/ipahelp/ipaprvw2.htm}{IPA
%     Help}.}

\newcommand{\skills}[2]{
  \item #2 #1
}

\vspace{-1em}

\section{Skills \& Languages}

\vspace{-1em}

\setlength{\columnsep}{-2cm}
\begin{multicols}{5}
\raggedcolumns
\begin{small}
\begin{itemize}
\renewcommand{\labelitemi}{}
\renewcommand{\skill}{\textnormal}
\setlength{\itemsep}{1pt}
\setlength{\parskip}{0pt}
\setlength{\parsep}{0pt}

\skills{\bash}{\threeskill}
\skills{\clojure}{\threeskill}
\skills{\git}{\threeskill}
\skills{GNU/\linux}{\threeskill}
\skills{\footnotesize{\hadoop ecosystem}}{\threeskill}
\skills{\haskell}{\oneskill}
\skills{\java}{\threeskill}
\skills{\js}{\twoskill}
\skills{\LaTeX}{\twoskill}
\skills{\skill{Nix/NixOS}}{\oneskill}
\skills{\postgres}{\threeskill}
\skills{\python}{\threeskill}
\skills{\scala}{\twoskill}
\skills{\scheme}{\threeskill}
\skills{\spark}{\threeskill}

% graveyard

%% \skills{\numpy/\scipy}{\threeskill}
%% \skills{\opencv}{\threeskill}
%% \skills{\django}{\twoskill}
%% \skills{\matlab}{\twoskill}
%% \skills{Blender}{\twoskill}
%% \skills{Emacs Lisp}{\twoskill}
%% \skills{English}{\threeskill}
%% \skills{LAMP Stack}{\fournotes}
%% \skills{LISP}{\onenote}
%% \skills{Learning}{\fournotes Machine}
%% \skills{MS Office}{\fivenotes}
%% \skills{Maple}{\twoskill}
%% \skills{Networking}{\threenotes}
%% \skills{Processing}{\fivenotes Image}
%% \skills{Sys. Admin.}{\threenotes}
%% \skills{Visual Basic}{\fivenotes}
%% \skills{Windows}{\fivenotes}
%% \skills{Wordpress}{\fournotes}
%% \skills{\ccpp}{\twoskill}
%% \skills{\hive}{\oneskill}
%% \skills{\html}{\threeskill}
%% \skills{\jquery}{\twoskill}
%% \skills{\php}{\oneskill}
%% \skills{\scheme}{\threeskill}
\end{itemize}
\end{small}
\end{multicols}
\setlength{\columnsep}{0pt}

\vspace{-1.5em}

\begin{footnotesize}
  \oneskill Small-scale or personal projects \hfill
  \twoskill Used in production  \hfill
  \threeskill Used in large-scale production systems
\end{footnotesize}

\newcommand{\proj}[3]{
  \textsc{#1} & #2\\
   &\href{http://www.#3}{#3}\\
   \multicolumn{2}{c}{} \\ [-1ex]
}

\newcommand{\projl}[3]{
  \textsc{#1} & #2\\
   &\href{http://www.#3}{#3}\\
}

\newcommand{\projlh}[4]{
  \textsc{#1} & #2\\
   &\href{#3}{#4}\\
}
%% \section{Personal and Open Source Projects}
%% \begin{tabularx}{\textwidth}{@{}p{3cm}|X@{}}

%%   \proj{matsciseg}%
%%   {Framework for propagated 3D volume segmentation, used in my
%%     dissertation work.  Algorithms created in \python and \cpp and
%%     exposed as a web API using \django. Includes a web application
%%     that consumes the API created in \js, and \jquery.}%
%%   {github.com/malloc47/matsciseg}

%% %%   \proj{\href{http://nonpartisan.me}{nonpartisan.me}}%
%% %% {Google Chrome extension that filters social media websites for political keywords.  Available in the \href{https://chrome.google.com/webstore/detail/nonpartisanme/ninebcppidndhampaggnjbijpacoadgg}{Chrome Web Store}.  Featured in the \href{http://www.charlestoncitypaper.com/charleston/sick-of-politics-on-facebook-try-this-browser-tool/Content?oid=4153447}{Charleston City Paper}.}%
%% %% {github.com/malloc47/nonpartisan.me}

%%   \proj{befunge.py}{Complete \href{https://en.wikipedia.org/wiki/Befunge}{Befunge} interpreter written in \python.  Implements the Befunge 93 specification, and is one of the closest Python equivalents to the \c reference implementation.}{github.com/malloc47/befunge.py}

%%   \proj{term-do}{An interactive terminal prompt that displays potential command completions as you type.  A hybrid of gnome-do and Emacs's ido-mode.  Works on many tested VT100 terminal types; built in~\cpp.  Includes client/server architecture implemented with boost.interprocess and full-featured plugin system.
%%     %Available in the \href{https://aur.archlinux.org/packages/term-do-git/}{Arch Linux AUR}.
%%   }{github.com/malloc47/term-do}

%%   %% \proj{Ratio Contour}{Maintainer and contributor for the Ratio Contour project, a salient object detection and segmentation method used for computer vision applications.  Developed in \c and \matlab.}{github.com/malloc47/ratio-contour}

%%   % \projl{PMLDAP}{\linux user management tool for Linux clusters.  Created as a simplified replacement for LDAP.  Capable of bootstrapping new systems, synchronizing users and configuration files, and running distributed commands.  Written in \bash.}{github.com/malloc47/pmldap}

%%   %% \projlh{Sina Weibo Mobile Client}{Created a \skill{J2ME}-based prototype mobile client for the popular Chinese \institution{Sina} microblogging service, similar to \institution{Twitter}.  Targeted at limited-functionality CLDC phones and uses a custom \java wrapper for the \institution{Sina} API.}{http://bd.weibo.10086.cn/2012/downloads_kjava}{bd.weibo.10086.cn/2012/downloads\_kjava}

%%   % \proj{Digital Collation}{Research prototype to ``collate'' high-resolution document scans using image registration.  Written in \matlab utilizing various computer vision libraries.}{www.github.com/malloc47/digital-collation}

%%   % \proj{matscicut}{An energy minimization framework for segmenting 3D materials volumes. Prototype of dissertation work, created in C++ using OpenCV libraries, with assorted MATLAB helper utilities.}{www.github.com/malloc47/matscicut}

%%   % \proj{git-hq}{A remote management system for git, coded in \python.}{www.github.com/malloc47/git-hq}

%% \end{tabularx}

%\section{Publications}
\let\originalbibitem\bibitem
\def\bibitem#1#2\par{%
  \noexpandarg
  \originalbibitem{#1}
  \StrSubstitute{#2}{Jarrell Waggoner}{\textbf{Jarrell Waggoner}}\par}

\nocite{derrick:16}
\nocite{waggoner:15}
\nocite{waggoner:phd}
\nocite{waggoner:14}
%% \nocite{zhou:14}
\nocite{waggoner:13a}
% \nocite{waggoner:13b}
%% \nocite{waggoner:13c}
%% \nocite{waggoner:12}
\nocite{barbu:12}
\nocite{zhang:12}
%% \nocite{wang:11}
\nocite{waggoner:11}
\nocite{temlyakov:10}
\nocite{zhang:10}
% \nocite{temlyakov:13}
% \nocite{salvi:13a}
% \nocite{salvi:13b}

\renewcommand\refname{Selected Publications}
{\footnotesize \bibliography{cv}}
\bibliographystyle{plainyr-rev}

\section{Recent Talks}
{\footnotesize
\begin{enumerate}[align=left,labelsep=0em]
\renewcommand{\labelenumi}{[\arabic{enumi}]}
\item \href{http://www.malloc47.com/posscon2015/}{Rules Engines: Logic As Data Structure}. \emph{\href{http://posscon.org/}{Palmetto Open Source Software Conference}}. Columbia, SC. April 14, 2015.
\item \href{http://www.malloc47.com/ato2013/}{Python for Computer Vision}. \emph{\href{http://allthingsopen.org/}{All Things Open}}. Raleigh, SC. October 24, 2013.
%% \item \href{http://www.malloc47.com/gsd2013/}{Interactive Grain Image Segmentation Using Graph Cut Algorithms}. \emph{USC Graduate Student Day}. Columbia, SC. April 12, 2013.
\item \href{http://www.malloc47.com/posscon2013/}{Extending Django}. \emph{\href{http://posscon.org/}{Palmetto Open Source Software Conference}}.  Columbia, SC.  March 28, 2013.
\item \href{http://www.malloc47.com/cs-careers/}{Computer Science: Research, Industry, and Entrepreneurship}.  \emph{Careers in Science Lecture Series}.  Lancaster, SC.  March 6, 2013.
%% \item \href{http://www.malloc47.com/spie2013/}{Interactive Grain Image Segmentation Using Graph Cut Algorithms}.  \emph{SPIE (Computational Imaging XI)}.  Burlingame, CA.  February 6, 2013.
%% \item Android Application Development Workshop.  \emph{Appathon Contest}.  Columbia, SC.  Nov. 17, 2012.
%% \item Open Source and Education. \emph{SC Municipal Technology Association (SCMTA) Conference}. Charleston, SC.  Sep. 6, 2012.
%% \item Open Source and Higher Education.  \emph{SC Technical College System (SCTCS) Conference}.  Columbia, SC.  Sep. 25, 2012.
%% \item Introduction to Android Development.  \emph{Digital Humanities High Performance Computing (DHHPC) Workshop}.  Columbia, SC.  Aug. 8, 2012.
%% \item Combining Global Labeling and Local Relabeling for Metallic Image Segmentation.  \emph{SPIE (Computational Imaging X)}. Jan. 23, 2012.
%% \item Open Source and Government.  \emph{SC Government Management Information Systems (SCGMIS) Workshop.}  Columbia, SC.  Jan. 19, 2012.
\end{enumerate} }

%% \section{Honors/Awards at USC}
%% \begin{tabularx}{\textwidth}{@{}r|X l|p{4.9cm}@{}}
%% 2012 & \small{Gamecock Computing Research Symposium Poster Session,  First Place} &
%% 2004 & Clara P. Hammond Award  \\

%% 2012 & Graduate Student Day Presentation,  First Place &
%% 2004 & Science and Mathematics Award \\

%% 2009 & Upsilon Pi Epsilon &
%% 2004 & Highest Academic Average Award \\
%% \end{tabularx}

\section{Activities}
teaching, open source software, GIS visualization, Linux,
\href{https://soundcloud.com/malloc47}{music composition}

\null\vfill
\footnotesize{
  Online:  \href{http://resume.malloc47.com}{resume.malloc47.com} \hfill
  Full CV: \href{http://cv.malloc47.com}{cv.malloc47.com} \hfill
  Source:  \href{https://github.com/malloc47/cv}{github.com/malloc47/cv/}
}

\pagestyle{myheadings}
\markright{Jarrell Waggoner}

%%\XeTeXpdffile ''resume.pdf'' page 1 scaled 800

\end{document}
